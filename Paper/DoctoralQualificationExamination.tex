% Options for packages loaded elsewhere
\PassOptionsToPackage{unicode}{hyperref}
\PassOptionsToPackage{hyphens}{url}
\PassOptionsToPackage{dvipsnames,svgnames,x11names}{xcolor}
%
\documentclass[
  12pt,
]{ctexart}
\title{政策扩散与政府间关系}
\author{孙宇飞\footnote{清华大学政治学系博士生,联系电话:18638750921,邮箱:\href{mailto:sunyf20@mails.tsinghua.edu.cn}{\nolinkurl{sunyf20@mails.tsinghua.edu.cn}}}}
\date{}

\usepackage{amsmath,amssymb}
\usepackage{lmodern}
\usepackage{iftex}
\ifPDFTeX
  \usepackage[T1]{fontenc}
  \usepackage[utf8]{inputenc}
  \usepackage{textcomp} % provide euro and other symbols
\else % if luatex or xetex
  \usepackage{unicode-math}
  \defaultfontfeatures{Scale=MatchLowercase}
  \defaultfontfeatures[\rmfamily]{Ligatures=TeX,Scale=1}
\fi
% Use upquote if available, for straight quotes in verbatim environments
\IfFileExists{upquote.sty}{\usepackage{upquote}}{}
\IfFileExists{microtype.sty}{% use microtype if available
  \usepackage[]{microtype}
  \UseMicrotypeSet[protrusion]{basicmath} % disable protrusion for tt fonts
}{}
\makeatletter
\@ifundefined{KOMAClassName}{% if non-KOMA class
  \IfFileExists{parskip.sty}{%
    \usepackage{parskip}
  }{% else
    \setlength{\parindent}{0pt}
    \setlength{\parskip}{6pt plus 2pt minus 1pt}}
}{% if KOMA class
  \KOMAoptions{parskip=half}}
\makeatother
\usepackage{xcolor}
\IfFileExists{xurl.sty}{\usepackage{xurl}}{} % add URL line breaks if available
\IfFileExists{bookmark.sty}{\usepackage{bookmark}}{\usepackage{hyperref}}
\hypersetup{
  pdftitle={政策扩散与政府间关系},
  pdfauthor={孙宇飞},
  colorlinks=true,
  linkcolor={Maroon},
  filecolor={Maroon},
  citecolor={Blue},
  urlcolor={Blue},
  pdfcreator={LaTeX via pandoc}}
\urlstyle{same} % disable monospaced font for URLs
\usepackage[margin=1in]{geometry}
\usepackage{longtable,booktabs,array}
\usepackage{calc} % for calculating minipage widths
% Correct order of tables after \paragraph or \subparagraph
\usepackage{etoolbox}
\makeatletter
\patchcmd\longtable{\par}{\if@noskipsec\mbox{}\fi\par}{}{}
\makeatother
% Allow footnotes in longtable head/foot
\IfFileExists{footnotehyper.sty}{\usepackage{footnotehyper}}{\usepackage{footnote}}
\makesavenoteenv{longtable}
\usepackage{graphicx}
\makeatletter
\def\maxwidth{\ifdim\Gin@nat@width>\linewidth\linewidth\else\Gin@nat@width\fi}
\def\maxheight{\ifdim\Gin@nat@height>\textheight\textheight\else\Gin@nat@height\fi}
\makeatother
% Scale images if necessary, so that they will not overflow the page
% margins by default, and it is still possible to overwrite the defaults
% using explicit options in \includegraphics[width, height, ...]{}
\setkeys{Gin}{width=\maxwidth,height=\maxheight,keepaspectratio}
% Set default figure placement to htbp
\makeatletter
\def\fps@figure{htbp}
\makeatother
\setlength{\emergencystretch}{3em} % prevent overfull lines
\providecommand{\tightlist}{%
  \setlength{\itemsep}{0pt}\setlength{\parskip}{0pt}}
\setcounter{secnumdepth}{5}
\newlength{\cslhangindent}
\setlength{\cslhangindent}{1.5em}
\newlength{\csllabelwidth}
\setlength{\csllabelwidth}{3em}
\newlength{\cslentryspacingunit} % times entry-spacing
\setlength{\cslentryspacingunit}{\parskip}
\newenvironment{CSLReferences}[2] % #1 hanging-ident, #2 entry spacing
 {% don't indent paragraphs
  \setlength{\parindent}{0pt}
  % turn on hanging indent if param 1 is 1
  \ifodd #1
  \let\oldpar\par
  \def\par{\hangindent=\cslhangindent\oldpar}
  \fi
  % set entry spacing
  \setlength{\parskip}{#2\cslentryspacingunit}
 }%
 {}
\usepackage{calc}
\newcommand{\CSLBlock}[1]{#1\hfill\break}
\newcommand{\CSLLeftMargin}[1]{\parbox[t]{\csllabelwidth}{#1}}
\newcommand{\CSLRightInline}[1]{\parbox[t]{\linewidth - \csllabelwidth}{#1}\break}
\newcommand{\CSLIndent}[1]{\hspace{\cslhangindent}#1}
\ifLuaTeX
  \usepackage{selnolig}  % disable illegal ligatures
\fi

\begin{document}
\maketitle

\hypertarget{ux653fux7b56ux6269ux6563ux4e0eux653fux5e9cux95f4ux5173ux7cfb}{%
\section{政策扩散与政府间关系}\label{ux653fux7b56ux6269ux6563ux4e0eux653fux5e9cux95f4ux5173ux7cfb}}

\begin{figure}
\includegraphics[width=1\linewidth]{../figures/分析框架} \caption{本文分析框架}\label{fig:unnamed-chunk-1}
\end{figure}

\hypertarget{ux6269ux6563ux6982ux5ff5}{%
\subsection{扩散概念}\label{ux6269ux6563ux6982ux5ff5}}

\hypertarget{ux6982ux5ff5ux754cux5b9a}{%
\subsubsection{概念界定}\label{ux6982ux5ff5ux754cux5b9a}}

\textbf{政策扩散的概念和内涵经历了一个从``单维''到``多维''的演变过程。}
在一个日益相互依存的治理环境中,扩散已成为政策传播的一个决定性特征。(\protect\hyperlink{ref-GilardiWasserfallen2019}{GILARDI 等, 2019})一个政治实体(国家、国际组织、地方政府等)采取的政策不仅会受到内部因素影响,还会受到外部行为者政策影响,这一过程通常被称为政策扩散。{[}GilardiEtAl2021a{]}。从 \protect\hyperlink{ref-Walker1969}{WALKER} (\protect\hyperlink{ref-Walker1969}{1969}) 提出政策扩散的概念开始的五十年,政治学和公共管理等学科领域对政策扩散的研究方兴未艾。同时,随着研究的不断深入和研究重点的转换,政策扩散的概念也在不断变化。

\textbf{政策创新视角。}政策扩散的早期研究基于政策过程理论展开(\protect\hyperlink{ref-BaoWeiHui2021}{鲍伟慧, 2021}),因此,这一阶段的政策扩散概念,更多的是和政策创新是一体两面,紧密相连。(\protect\hyperlink{ref-Walker1969}{WALKER, 1969})基于政策创新视角,学者们更加注重政策扩散中的首次使用, \protect\hyperlink{ref-Walker1969}{WALKER} (\protect\hyperlink{ref-Walker1969}{1969}) 将政策扩散定义为某个政府首次采纳某项政策的行为,无论这个政或项目被提出多长时间,只要被内部行为者吸纳,即为政策扩散。 \protect\hyperlink{ref-Lucas1983}{LUCAS} (\protect\hyperlink{ref-Lucas1983}{1983}) 对政策扩散的定义虽然更加侧重于政策的执行而非首次出台,但是``创新''在其政策扩散的特点仍十分突出,他认为政策扩散是从首创者流向其他政府部门的现象,外来政策被当地政府首次接受并执行即为政策扩散。他还强调组织对于政策扩散的影响,他认为政策创新的扩散是在组织中传递的,又推动者组织的变革。这一前瞻性的定义为之后注重政策过程和结果的政策扩散定义打下基础。

\textbf{政策过程视角。}对于政策创新的重视,虽然有利于用现有的理论对这一现象进行解释,但也会导致支持创新的过分关注,即专注于采用创新而排除传播和政策制定的其他潜在重要特征的趋势,从而使我们无法更广泛地了解这些过程(\protect\hyperlink{ref-GilardiEtAl2021}{GILARDI 等, 2021} ; \protect\hyperlink{ref-Rogers2003}{ROGERS, 2003} ; \protect\hyperlink{ref-Karch2007}{KARCH, 2007})。随着政策扩散研究的进一步深入,学者们对政策扩散理解的注重点``首次采纳''拓展至``政策过程'',更多的从内外部行为者之间的互动过程角度理解政策扩散。在这一互动过程中,沟通交流和组织对政策扩散的影响尤为重要。。\protect\hyperlink{ref-Rogers2003}{ROGERS} (\protect\hyperlink{ref-Rogers2003}{2003}) 将政策扩散从首次采纳,定义为``互动-采纳-治理''的政策过程中的创新扩散。进一步的,\protect\hyperlink{ref-GilardiEtAl2021}{GILARDI 等} (\protect\hyperlink{ref-GilardiEtAl2021}{2021}) 从问题定义和议程设置视角定义政策扩散,将政策扩散的研究再次拓展到政策制定的整个过程。

\textbf{政策结果视角。}对如何扩散的过度关注是政策扩散研究受到的主要批评之一。作为对这一批评的回应,学者开始从结果角度理解政策扩散。这一方面的研究主要包括政策趋同(Policy Convergence)和政策再造(Policy Reinvention)两个方面。\protect\hyperlink{ref-Berry1994}{BERRY} (\protect\hyperlink{ref-Berry1994}{1994}) 将政策扩散的过程定义为在不同的地理空间,某一方面的政策的相似性增加。\protect\hyperlink{ref-Inkeles2019}{INKELES} (\protect\hyperlink{ref-Inkeles2019}{2019}) 将政策扩散定义为政府政策从不同的位置,人为的变化到某些同一位置。除了政策趋同之外,学者们还从政策扩散的差异结果研究政策扩散,并提出政策再造的概念(\protect\hyperlink{ref-Clark1985}{CLARK, 1985})。政策扩散并不一定会导致不同部门政策的完全相同,政策扩散的对象即内部行动者也不是完全被动的接受政策扩散,反而会根据自身的实际情况对政策进行批判性接受(\protect\hyperlink{ref-GlickHays1991}{GLICK 等, 1991}; \protect\hyperlink{ref-Hays1996}{HAYS, 1996}; \protect\hyperlink{ref-MooneyLee1995}{MOONEY 等, 1995})。还有学者从政策执行的角度关注政策扩散,认为政策扩散是指某一政府部门的政策影响到其他政府部门的治理过程。(\protect\hyperlink{ref-Evans2009}{EVANS, 2009})

\textbf{全球化的视角}虽然政策扩散的概念起始于美国政治研究(\protect\hyperlink{ref-Walker1969}{WALKER, 1969}),但是受到比较政治学和国际政治领域学者的关注。(\protect\hyperlink{ref-Milner1998}{MILNER, 1998})国际关系和比较政治的学者们将政策扩散从国内政治领域拓展到国际政治层面。与比较政治文献一样,国际关系学者一直关注趋同,但国际关系学者更加注重国际组织在促进各国实现相似政策方面的作用,尤其是关于规范的扩散。这些研究深入考察了社会化过程和身份政治如何影响规范在国际社会的传播,尽管这些概念与美国地方政治有关,但在国内政策扩散的研究中并未探讨这些概念。(\protect\hyperlink{ref-Checkel1999}{CHECKEL, 1999})

根据上述梳理,我们可以发现,政策扩散的定义随着研究的不断深入而越来越丰富。从政策扩散的主体来看,从政府拓展到各类包括国际组织、社团等各类政治主体;从政策扩散的过程来看,从单一的政策首次接纳,扩展到``议程设置-政策采用-政策执行-政策结果''的整个过程;从政策扩散的内容来看,政策扩散从单一的国内成文政策扩展到全球政策和规范;从研究领域来看,政策扩散最早由美国国内政治的研究者提出后,迅速的被比较政治学、国际关系等领域的学者接受和借用。

\begin{figure}
\includegraphics[width=1\linewidth]{../figures/ngram} \caption{政策扩散的定义}\label{fig:unnamed-chunk-2}
\end{figure}

\hypertarget{ux6982ux5ff5ux6f14ux53d8}{%
\subsubsection{概念演变}\label{ux6982ux5ff5ux6f14ux53d8}}

\begin{figure}
\includegraphics[width=1\linewidth]{../figures/政策扩散的概念演变} \caption{政策扩散的概念演变}\label{fig:unnamed-chunk-3}
\end{figure}

\hypertarget{ux4e34ux8fd1ux6982ux5ff5ux8fa8ux6790ux653fux7b56ux8f6cux79fbux653fux7b56ux8d8bux540c}{%
\subsubsection{临近概念辨析:政策转移、政策趋同}\label{ux4e34ux8fd1ux6982ux5ff5ux8fa8ux6790ux653fux7b56ux8f6cux79fbux653fux7b56ux8d8bux540c}}

\hypertarget{ux6982ux5ff5ux610fux4e49}{%
\subsubsection{概念意义}\label{ux6982ux5ff5ux610fux4e49}}

\hypertarget{ux653fux7b56ux6269ux6563ux7814ux7a76ux7684ux5b66ux79d1ux8303ux5f0f}{%
\subsubsection{政策扩散研究的学科范式}\label{ux653fux7b56ux6269ux6563ux7814ux7a76ux7684ux5b66ux79d1ux8303ux5f0f}}

代表性和特殊性

\url{https://kns.cnki.net/KXReader/Detail?invoice=r5ANsDAUmdDf8Bk5cQ4QEkYYI\%2FepadMtp94e3AZJ0Y1MZyzfQH\%2B19h5UftFYFK9G2WJsYIrNoyGZqAII4I\%2FwO7dXp\%2Bujg0xOOTTRVgXZ3ap8NMVsaETJjN853shp\%2Fadfuulp\%2Fo9m8P3ZXGOR4Px6K91EhT97d6t\%2BA\%2F8HNt2zS2I\%3D\&DBCODE=CJFD\&FileName=XMDS201306002\&TABLEName=cjfdhis2\&nonce=33AEDC5208E04324A82FCB1E0CC3D374\&uid=\&TIMESTAMP=1638760166703\#65}

\url{https://kns.cnki.net/KXReader/Detail?invoice=PtzcpYTeZJt3BcL8PzJE0\%2BZ8orfspY57JJpEAf3nnZvRQpFnakSGduOZSJ7wwjv99lmeQPLMNeOQS0DVzi5xKD4\%2FnhwqT\%2FVodWvMIFRIFziZZf8LZvktdoEPWXurxhA\%2BOdxIW68nwkYnoUPPw30Vr5cY7X6iLKJ4Oawdwi7Nfog\%3D\&DBCODE=CJFD\&FileName=NMDB202104012\&TABLEName=cjfdlast2021\&nonce=7BACDB3113F04AF68241D968A50BDF34\&uid=\&TIMESTAMP=1638760387468}

\begin{itemize}
\item
  政策扩散与政策转移,政策学习
\item
  政策扩散与技术扩散
\end{itemize}

\hypertarget{ux653fux7b56ux6269ux6563ux4ee3ux8868ux6027ux548cux7279ux6b8aux6027}{%
\subsubsection{政策扩散代表性和特殊性}\label{ux653fux7b56ux6269ux6563ux4ee3ux8868ux6027ux548cux7279ux6b8aux6027}}

\begin{itemize}
\tightlist
\item
  政策扩散包含decision maker, policy and policy implication
\item
  政策扩散兼具时效性和长期性
\end{itemize}

\hypertarget{ux653fux7b56ux6269ux6563ux7814ux7a76ux7684ux805aux7c7bux5206ux6790}{%
\subsubsection{政策扩散研究的聚类分析}\label{ux653fux7b56ux6269ux6563ux7814ux7a76ux7684ux805aux7c7bux5206ux6790}}

\hypertarget{ux6269ux6563ux4e3bux4f53}{%
\subsection{扩散主体}\label{ux6269ux6563ux4e3bux4f53}}

\hypertarget{ux5185ux90e8ux4e3bux4f53}{%
\subsubsection{内部主体}\label{ux5185ux90e8ux4e3bux4f53}}

\begin{itemize}
\item
  内部行动者的类别
\item
  内部行动者如何影响政策扩散

  \begin{itemize}
  \item
    偏好
  \item
    目标
  \item
    能力
  \item
    和其他行动者的互动
  \end{itemize}
\end{itemize}

\hypertarget{ux5916ux90e8ux4e3bux4f53}{%
\subsubsection{外部主体}\label{ux5916ux90e8ux4e3bux4f53}}

\begin{itemize}
\item
  外部行动者的类别

  \begin{itemize}
  \item
    中央政府
  \item
    其他地方政府
  \item
    其他国家和超国家组织
  \item
    政策企业家
  \end{itemize}
\end{itemize}

\hypertarget{ux6269ux6563ux5185ux5bb9}{%
\subsection{扩散内容}\label{ux6269ux6563ux5185ux5bb9}}

\hypertarget{ux5e38ux89c4ux653fux7b56ux6269ux6563}{%
\subsubsection{常规政策扩散}\label{ux5e38ux89c4ux653fux7b56ux6269ux6563}}

\begin{itemize}
\item
  福利政策
\item
  教育政策
\item
  公民权利政策
\end{itemize}

\hypertarget{ux975eux5e38ux89c4ux653fux7b56ux7684ux6269ux6563}{%
\subsubsection{非常规政策的扩散}\label{ux975eux5e38ux89c4ux653fux7b56ux7684ux6269ux6563}}

\begin{itemize}
\item
  制度扩散
\item
  体制扩散
\item
  骚乱和政变的扩散
\end{itemize}

\hypertarget{ux6269ux6563ux903bux8f91}{%
\subsection{扩散逻辑}\label{ux6269ux6563ux903bux8f91}}

这部分分类不好,不够理论,看我改的,再做修改

\hypertarget{ux7b56ux7565ux6027ux6269ux6563strategic-action}{%
\subsubsection{策略性扩散(Strategic action)}\label{ux7b56ux7565ux6027ux6269ux6563strategic-action}}

\hypertarget{ux65f6ux6548ux6027ux6269ux6563common-shock}{%
\subsubsection{时效性扩散(Common shock)}\label{ux65f6ux6548ux6027ux6269ux6563common-shock}}

\hypertarget{ux76f8ux4f3cux5bfcux81f4ux6269ux6563homophily}{%
\subsubsection{相似导致扩散(Homophily)}\label{ux76f8ux4f3cux5bfcux81f4ux6269ux6563homophily}}

\hypertarget{ux6269ux6563ux8defux5f84-ux8981ux548cux4e0aux9762ux505aux597dux533aux5206}{%
\subsection{扩散路径 (要和上面做好区分)}\label{ux6269ux6563ux8defux5f84-ux8981ux548cux4e0aux9762ux505aux597dux533aux5206}}

\hypertarget{ux5b66ux4e60ux673aux5236}{%
\subsubsection{学习机制}\label{ux5b66ux4e60ux673aux5236}}

\begin{itemize}
\item
  谁来学

  \begin{itemize}
  \item
    横向的学习
  \item
    纵向的学习
  \end{itemize}
\item
  学什么

  \begin{itemize}
  \item
    正向学习
  \item
    负向学习
  \end{itemize}
\item
  怎么学

  \begin{itemize}
  \item
    什么是成功的政策
  \item
    如何识别政策成功
  \end{itemize}
\end{itemize}

\hypertarget{ux7adeux4e89ux673aux5236}{%
\subsubsection{竞争机制}\label{ux7adeux4e89ux673aux5236}}

\begin{itemize}
\item
  怎么竞争
\item
  竞争的后果

  \begin{itemize}
  \item
    竞争的正向效应
  \item
    负向的负向效应
  \end{itemize}
\end{itemize}

\hypertarget{ux5f3aux5236ux673aux5236}{%
\subsubsection{强制机制}\label{ux5f3aux5236ux673aux5236}}

\begin{itemize}
\item
  谁强制,强制谁
\item
  怎么强制:强制的手段
\end{itemize}

\hypertarget{ux793eux4f1aux5316ux673aux5236}{%
\subsubsection{社会化机制}\label{ux793eux4f1aux5316ux673aux5236}}

\begin{itemize}
\item
  谁来社会化

  \begin{itemize}
  \item
    社会化的行动者
  \item
    不是单向的
  \end{itemize}
\item
  怎么社会化
\item
  社会化的效应(特点)
\end{itemize}

\hypertarget{ux591aux5143ux673aux5236}{%
\subsubsection{多元机制}\label{ux591aux5143ux673aux5236}}

\begin{itemize}
\item
  多种机制混合
\item
  补充而非替代
\end{itemize}

\hypertarget{ux653fux7b56ux6269ux6563ux65b9ux6cd5ux8defux5f84}{%
\subsection{政策扩散方法路径}\label{ux653fux7b56ux6269ux6563ux65b9ux6cd5ux8defux5f84}}

\hypertarget{ux73b0ux8c61ux5206ux6790}{%
\subsubsection{现象分析}\label{ux73b0ux8c61ux5206ux6790}}

\hypertarget{ux673aux5236ux5206ux6790}{%
\subsubsection{机制分析}\label{ux673aux5236ux5206ux6790}}

\hypertarget{ux6df7ux5408ux5206ux6790}{%
\subsubsection{混合分析}\label{ux6df7ux5408ux5206ux6790}}

\hypertarget{ux73b0ux6709ux4e0dux8db3ux4e0eux76f2ux70b9ux8fd9ux4e00ux6574ux5757ux6574ux5408ux5ea6ux5f88ux4f4e}{%
\section{现有不足与盲点(这一整块整合度很低)}\label{ux73b0ux6709ux4e0dux8db3ux4e0eux76f2ux70b9ux8fd9ux4e00ux6574ux5757ux6574ux5408ux5ea6ux5f88ux4f4e}}

\hypertarget{ux89e3ux91caux7684ux4e0dux8db3}{%
\subsection{解释的不足}\label{ux89e3ux91caux7684ux4e0dux8db3}}

\hypertarget{ux8c01ux6269ux6563}{%
\subsubsection{谁扩散}\label{ux8c01ux6269ux6563}}

\begin{itemize}
\item
  内部参与者

  \begin{itemize}
  \item
    对决策者的联系关注不足(引出后文的干部交流)
  \item
    把决策者异质性关注的不足(引出后文的能力)
  \item
    对执行者关注的不足(引出后文的执行者)
  \end{itemize}
\item
  外部参与者

  \begin{itemize}
  \item
    对民众关注不足(引出后文的回应之回应)
  \item
    对全球化的关注不足(引出后文的国际组织直接影响地方政府)
  \end{itemize}
\item
  国际和地方研究的脱节
\end{itemize}

\hypertarget{ux6269ux6563ux4ec0ux4e48}{%
\subsubsection{扩散什么}\label{ux6269ux6563ux4ec0ux4e48}}

\begin{itemize}
\item
  政策扩散使用二分的观点,对文本变化的关注不足(引出后文的政策再造)
\item
  过于关注政策采用而非政策制定(引出议程设置中的政策扩散,``会议政治'')
\item
  过于关注首次采用而非一个多次的互动方式
\item
  对政策本身关注的不足
\item
  对政策执行中的政策扩散关注不足
\end{itemize}

\hypertarget{ux4e3aux4ec0ux4e48ux6269ux6563}{%
\subsubsection{为什么扩散}\label{ux4e3aux4ec0ux4e48ux6269ux6563}}

\begin{itemize}
\item
  对强制的自主性关注不足
\item
  对机制的整合不足
\end{itemize}

\hypertarget{ux600eux4e48ux6269ux6563}{%
\subsubsection{怎么扩散}\label{ux600eux4e48ux6269ux6563}}

\hypertarget{ux4f7fux7528ux7684ux4e0dux8db3}{%
\subsection{使用的不足}\label{ux4f7fux7528ux7684ux4e0dux8db3}}

\hypertarget{ux65b9ux6cd5ux7684ux4e0dux8db3}{%
\subsection{方法的不足}\label{ux65b9ux6cd5ux7684ux4e0dux8db3}}

\hypertarget{ux672aux6765ux65b9ux5411ux57faux4e8eux653fux6cbbux5b66ux89c6ux89d2-ux8fd9ux4e09ux90e8ux5206ux90fdux770bux4e0dux51faux5bf9ux653fux6cbbux5b66ux7684ux91cdux8981ux6027}{%
\section{未来方向:基于政治学视角 (这三部分都看不出对政治学的重要性)}\label{ux672aux6765ux65b9ux5411ux57faux4e8eux653fux6cbbux5b66ux89c6ux89d2-ux8fd9ux4e09ux90e8ux5206ux90fdux770bux4e0dux51faux5bf9ux653fux6cbbux5b66ux7684ux91cdux8981ux6027}}

\hypertarget{ux4f5cux4e3aux56e0ux53d8ux91cfux7684ux653fux7b56ux6269ux6563}{%
\subsection{作为因变量的政策扩散}\label{ux4f5cux4e3aux56e0ux53d8ux91cfux7684ux653fux7b56ux6269ux6563}}

\hypertarget{ux56fdux5bb6ux80fdux529bux4e0eux653fux7b56ux6269ux6563ux7814ux7a76}{%
\subsubsection{国家能力与政策扩散研究}\label{ux56fdux5bb6ux80fdux529bux4e0eux653fux7b56ux6269ux6563ux7814ux7a76}}

\hypertarget{ux653fux5e9cux56deux5e94ux6027ux4e0eux653fux7b56ux6269ux6563ux7814ux7a76}{%
\subsubsection{政府回应性与政策扩散研究}\label{ux653fux5e9cux56deux5e94ux6027ux4e0eux653fux7b56ux6269ux6563ux7814ux7a76}}

\hypertarget{ux5168ux7403ux5316ux4e0eux653fux7b56ux6269ux6563ux7814ux7a76}{%
\subsubsection{全球化与政策扩散研究}\label{ux5168ux7403ux5316ux4e0eux653fux7b56ux6269ux6563ux7814ux7a76}}

\hypertarget{ux4f5cux4e3aux81eaux53d8ux91cfux7684ux653fux7b56ux6269ux6563}{%
\subsection{作为自变量的政策扩散}\label{ux4f5cux4e3aux81eaux53d8ux91cfux7684ux653fux7b56ux6269ux6563}}

\hypertarget{ux89c2ux5bdfux7eb5ux5411ux653fux5e9cux95f4ux5173ux7cfbux7684ux7a97ux53e3}{%
\subsubsection{观察纵向政府间关系的窗口}\label{ux89c2ux5bdfux7eb5ux5411ux653fux5e9cux95f4ux5173ux7cfbux7684ux7a97ux53e3}}

\hypertarget{ux89c2ux5bdfux6a2aux5411ux653fux5e9cux95f4ux5173ux7cfbux7684ux7a97ux53e3}{%
\subsubsection{观察横向政府间关系的窗口}\label{ux89c2ux5bdfux6a2aux5411ux653fux5e9cux95f4ux5173ux7cfbux7684ux7a97ux53e3}}

\hypertarget{ux89c2ux5bdfux56fdux5bb6ux793eux4f1aux5173ux7cfbux7684ux7a97ux53e3}{%
\subsubsection{观察国家社会关系的窗口}\label{ux89c2ux5bdfux56fdux5bb6ux793eux4f1aux5173ux7cfbux7684ux7a97ux53e3}}

\hypertarget{ux57faux4e8eux5927ux6570ux636eux65b9ux6cd5ux7684ux653fux7b56ux6269ux6563ux7814ux7a76}{%
\subsection{基于大数据方法的政策扩散研究}\label{ux57faux4e8eux5927ux6570ux636eux65b9ux6cd5ux7684ux653fux7b56ux6269ux6563ux7814ux7a76}}

\hypertarget{ux81eaux7136ux8bedux8a00ux5904ux7406ux65b9ux6cd5}{%
\subsubsection{自然语言处理方法}\label{ux81eaux7136ux8bedux8a00ux5904ux7406ux65b9ux6cd5}}

\hypertarget{ux793eux4ea4ux7f51ux7edcux5206ux6790ux65b9ux6cd5}{%
\subsubsection*{社交网络分析方法}\label{ux793eux4ea4ux7f51ux7edcux5206ux6790ux65b9ux6cd5}}
\addcontentsline{toc}{subsubsection}{社交网络分析方法}

\hypertarget{refs}{}
\begin{CSLReferences}{1}{0}
\leavevmode\vadjust pre{\hypertarget{ref-Berry1994}{}}%
BERRY F S, 1994. Sizing up State Policy Innovation Research{[}J{]}. Policy Studies Journal, 22(3): 442--456.

\leavevmode\vadjust pre{\hypertarget{ref-Checkel1999}{}}%
CHECKEL J T, 1999. Norms, Institutions, and National Identity in Contemporary {Europe}{[}J{]}. International studies quarterly, 43(1): 83--114.

\leavevmode\vadjust pre{\hypertarget{ref-Clark1985}{}}%
CLARK K B, 1985. The Interaction of Design Hierarchies and Market Concepts in Technological Evolution{[}J{]}. Research Policy, 14(5): 235--251. DOI:\href{https://doi.org/10.1016/0048-7333(85)90007-1}{10.1016/0048-7333(85)90007-1}.

\leavevmode\vadjust pre{\hypertarget{ref-Evans2009}{}}%
EVANS M, 2009. Policy Transfer in Critical Perspective{[}J{]}. Policy Studies, 30(3): 243--268. DOI:\href{https://doi.org/10.1080/01442870902863828}{10.1080/01442870902863828}.

\leavevmode\vadjust pre{\hypertarget{ref-GilardiEtAl2021}{}}%
GILARDI F, SHIPAN C R, WUEST B, 2021. Policy {Diffusion}: {The Issue-Definition Stage}{[}J{]}. AMERICAN JOURNAL OF POLITICAL SCIENCE, 65(1): 21--35. DOI:\href{https://doi.org/10.1111/ajps.12521}{10.1111/ajps.12521}.

\leavevmode\vadjust pre{\hypertarget{ref-GilardiWasserfallen2019}{}}%
GILARDI F, WASSERFALLEN F, 2019. The Politics of Policy Diffusion{[}J{]}. European Journal of Political Research, 58(4): 1245--1256. DOI:\href{https://doi.org/10.1111/1475-6765.12326}{10.1111/1475-6765.12326}.

\leavevmode\vadjust pre{\hypertarget{ref-GlickHays1991}{}}%
GLICK H R, HAYS S P, 1991. Innovation and {Reinvention} in {State Policymaking}: {Theory} and the {Evolution} of {Living Will Laws}{[}J{]}. The Journal of Politics, 53(3): 835--850. DOI:\href{https://doi.org/10.2307/2131581}{10.2307/2131581}.

\leavevmode\vadjust pre{\hypertarget{ref-Hays1996}{}}%
HAYS S P, 1996. Patterns of {Reinvention}{[}J{]}. Policy Studies Journal, 24(4): 551--566. DOI:\href{https://doi.org/10.1111/j.1541-0072.1996.tb01646.x}{10.1111/j.1541-0072.1996.tb01646.x}.

\leavevmode\vadjust pre{\hypertarget{ref-Inkeles2019}{}}%
INKELES A, 2019. One {World Emerging}? {Convergence And Divergence In Industrial Societies}{[}M{]}. {Routledge}.

\leavevmode\vadjust pre{\hypertarget{ref-Karch2007}{}}%
KARCH A, 2007. Democratic {Laboratories}: {Policy Diffusion Among} the {American States}{[}M{]}. {University of Michigan Press}.

\leavevmode\vadjust pre{\hypertarget{ref-Lucas1983}{}}%
LUCAS A, 1983. Public {Policy Diffusion Research}: {Integrating Analytic Paradigms}{[}J{]}. Knowledge, 4(3): 379--408. DOI:\href{https://doi.org/10.1177/107554708300400303}{10.1177/107554708300400303}.

\leavevmode\vadjust pre{\hypertarget{ref-Milner1998}{}}%
MILNER H V, 1998. Rationalizing Politics: {The} Emerging Synthesis of International, {American}, and Comparative Politics{[}J{]}. International Organization, 52(4): 759--786.

\leavevmode\vadjust pre{\hypertarget{ref-MooneyLee1995}{}}%
MOONEY C Z, LEE M-H, 1995. Legislative {Morality} in the {American States}: {The Case} of {Pre-Roe Abortion Regulation Reform}{[}J{]}. American Journal of Political Science, 39(3): 599--627. DOI:\href{https://doi.org/10.2307/2111646}{10.2307/2111646}.

\leavevmode\vadjust pre{\hypertarget{ref-Rogers2003}{}}%
ROGERS E M, 2003. Diffusion of {Innovations}, 5th {Edition}{[}M{]}. 5th edition 版. {New York}: {Free Press}.

\leavevmode\vadjust pre{\hypertarget{ref-Walker1969}{}}%
WALKER J L, 1969. The {Diffusion} of {Innovations} Among the {American States}{[}J{]}. American Political Science Review, 63(3): 880--899. DOI:\href{https://doi.org/10.1017/S0003055400258644}{10.1017/S0003055400258644}.

\leavevmode\vadjust pre{\hypertarget{ref-BaoWeiHui2021}{}}%
鲍伟慧, 2021. {政策扩散理论国外研究述评:态势、关注与展望}{[}J{]}. 内蒙古大学学报(哲学社会科学版), 53(04): 82--89.

\end{CSLReferences}

\end{document}
