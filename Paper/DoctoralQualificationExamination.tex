% Options for packages loaded elsewhere
\PassOptionsToPackage{unicode}{hyperref}
\PassOptionsToPackage{hyphens}{url}
\PassOptionsToPackage{dvipsnames,svgnames,x11names}{xcolor}
%
\documentclass[
  12pt,
]{ctexart}
\title{政策扩散与政府间关系}
\author{孙宇飞\footnote{清华大学政治学系博士生,联系电话:18638750921,邮箱:\href{mailto:sunyf20@mails.tsinghua.edu.cn}{\nolinkurl{sunyf20@mails.tsinghua.edu.cn}}}}
\date{}

\usepackage{amsmath,amssymb}
\usepackage{lmodern}
\usepackage{iftex}
\ifPDFTeX
  \usepackage[T1]{fontenc}
  \usepackage[utf8]{inputenc}
  \usepackage{textcomp} % provide euro and other symbols
\else % if luatex or xetex
  \usepackage{unicode-math}
  \defaultfontfeatures{Scale=MatchLowercase}
  \defaultfontfeatures[\rmfamily]{Ligatures=TeX,Scale=1}
\fi
% Use upquote if available, for straight quotes in verbatim environments
\IfFileExists{upquote.sty}{\usepackage{upquote}}{}
\IfFileExists{microtype.sty}{% use microtype if available
  \usepackage[]{microtype}
  \UseMicrotypeSet[protrusion]{basicmath} % disable protrusion for tt fonts
}{}
\makeatletter
\@ifundefined{KOMAClassName}{% if non-KOMA class
  \IfFileExists{parskip.sty}{%
    \usepackage{parskip}
  }{% else
    \setlength{\parindent}{0pt}
    \setlength{\parskip}{6pt plus 2pt minus 1pt}}
}{% if KOMA class
  \KOMAoptions{parskip=half}}
\makeatother
\usepackage{xcolor}
\IfFileExists{xurl.sty}{\usepackage{xurl}}{} % add URL line breaks if available
\IfFileExists{bookmark.sty}{\usepackage{bookmark}}{\usepackage{hyperref}}
\hypersetup{
  pdftitle={政策扩散与政府间关系},
  pdfauthor={孙宇飞},
  colorlinks=true,
  linkcolor={Maroon},
  filecolor={Maroon},
  citecolor={Blue},
  urlcolor={Blue},
  pdfcreator={LaTeX via pandoc}}
\urlstyle{same} % disable monospaced font for URLs
\usepackage[margin=1in]{geometry}
\usepackage{longtable,booktabs,array}
\usepackage{calc} % for calculating minipage widths
% Correct order of tables after \paragraph or \subparagraph
\usepackage{etoolbox}
\makeatletter
\patchcmd\longtable{\par}{\if@noskipsec\mbox{}\fi\par}{}{}
\makeatother
% Allow footnotes in longtable head/foot
\IfFileExists{footnotehyper.sty}{\usepackage{footnotehyper}}{\usepackage{footnote}}
\makesavenoteenv{longtable}
\usepackage{graphicx}
\makeatletter
\def\maxwidth{\ifdim\Gin@nat@width>\linewidth\linewidth\else\Gin@nat@width\fi}
\def\maxheight{\ifdim\Gin@nat@height>\textheight\textheight\else\Gin@nat@height\fi}
\makeatother
% Scale images if necessary, so that they will not overflow the page
% margins by default, and it is still possible to overwrite the defaults
% using explicit options in \includegraphics[width, height, ...]{}
\setkeys{Gin}{width=\maxwidth,height=\maxheight,keepaspectratio}
% Set default figure placement to htbp
\makeatletter
\def\fps@figure{htbp}
\makeatother
\setlength{\emergencystretch}{3em} % prevent overfull lines
\providecommand{\tightlist}{%
  \setlength{\itemsep}{0pt}\setlength{\parskip}{0pt}}
\setcounter{secnumdepth}{5}
\newlength{\cslhangindent}
\setlength{\cslhangindent}{1.5em}
\newlength{\csllabelwidth}
\setlength{\csllabelwidth}{3em}
\newlength{\cslentryspacingunit} % times entry-spacing
\setlength{\cslentryspacingunit}{\parskip}
\newenvironment{CSLReferences}[2] % #1 hanging-ident, #2 entry spacing
 {% don't indent paragraphs
  \setlength{\parindent}{0pt}
  % turn on hanging indent if param 1 is 1
  \ifodd #1
  \let\oldpar\par
  \def\par{\hangindent=\cslhangindent\oldpar}
  \fi
  % set entry spacing
  \setlength{\parskip}{#2\cslentryspacingunit}
 }%
 {}
\usepackage{calc}
\newcommand{\CSLBlock}[1]{#1\hfill\break}
\newcommand{\CSLLeftMargin}[1]{\parbox[t]{\csllabelwidth}{#1}}
\newcommand{\CSLRightInline}[1]{\parbox[t]{\linewidth - \csllabelwidth}{#1}\break}
\newcommand{\CSLIndent}[1]{\hspace{\cslhangindent}#1}
\ifLuaTeX
  \usepackage{selnolig}  % disable illegal ligatures
\fi

\begin{document}
\maketitle

当国家或民族采用新政策时,他们采用的决定不仅会受到内部因素的影响,还会受到外部因素的影响,这一过程通常被称为政策扩散。然而,如果政策确实扩散,它们不会直接从一个地方的采用传播到另一个地方的采用,正如大多数研究暗示的那样。相反,路径将从一个地方的采用流向政策过程的开始------问题定义阶段------在另一个地方。毕竟,政策制定分几个阶段进行,从问题的识别和定义开始,然后才(可能)最终通过。

政策扩散的定义,政策扩散的重要性,今天的世界以前所未有的方式相互联系,这些联系构成了地方、区域、州、国家和国际各级决策者面临的政策机遇和限制。

是什么、扩散什么、谁扩散、为什么扩散、怎么扩散、谁研究扩散、怎么研究扩散

\begin{figure}
\includegraphics[width=1\linewidth]{../figures/分析框架} \caption{本文分析框架}\label{fig:unnamed-chunk-1}
\end{figure}

\hypertarget{ux6269ux6563ux6982ux5ff5}{%
\section{扩散概念}\label{ux6269ux6563ux6982ux5ff5}}

政策扩散概念的意义和价值

\hypertarget{ux6982ux5ff5ux754cux5b9a}{%
\subsection{概念界定}\label{ux6982ux5ff5ux754cux5b9a}}

\textbf{政策扩散的概念和内涵经历了一个从``单维''到``多维''的演变过程。}
在一个日益相互依存的治理环境中,扩散已成为政策传播的一个决定性特征。(\protect\hyperlink{ref-GilardiWasserfallen2019}{GILARDI 等, 2019})一个政治实体(国家、国际组织、地方政府等)采取的政策不仅会受到内部因素影响,还会受到外部行为者政策影响,这一过程通常被称为政策扩散。{[}GilardiEtAl2021a{]}。从 \protect\hyperlink{ref-Walker1969}{WALKER} (\protect\hyperlink{ref-Walker1969}{1969}) 提出政策扩散的概念开始的五十年,政治学和公共管理等学科领域对政策扩散的研究方兴未艾。同时,随着研究的不断深入和研究重点的转换,政策扩散的概念也在不断变化。

\textbf{政策创新视角。}政策扩散的早期研究基于政策过程理论展开(\protect\hyperlink{ref-BaoWeiHui2021}{鲍伟慧, 2021}),因此,这一阶段的政策扩散概念,更多的是和政策创新是一体两面,紧密相连。(\protect\hyperlink{ref-Walker1969}{WALKER, 1969})基于政策创新视角,学者们更加注重政策扩散中的首次使用, \protect\hyperlink{ref-Walker1969}{WALKER} (\protect\hyperlink{ref-Walker1969}{1969}) 将政策扩散定义为某个政府首次采纳某项政策的行为,无论这个政或项目被提出多长时间,只要被内部行为者吸纳,即为政策扩散。 \protect\hyperlink{ref-Lucas1983}{LUCAS} (\protect\hyperlink{ref-Lucas1983}{1983}) 对政策扩散的定义虽然更加侧重于政策的执行而非首次出台,但是``创新''在其政策扩散的特点仍十分突出,他认为政策扩散是从首创者流向其他政府部门的现象,外来政策被当地政府首次接受并执行即为政策扩散。他还强调组织对于政策扩散的影响,他认为政策创新的扩散是在组织中传递的,又推动者组织的变革。这一前瞻性的定义为之后注重政策过程和结果的政策扩散定义打下基础。

\textbf{政策过程视角。}对于政策创新的重视,虽然有利于用现有的理论对这一现象进行解释,但也会导致支持创新的过分关注,即专注于采用创新而排除传播和政策制定的其他潜在重要特征的趋势,从而使我们无法更广泛地了解这些过程(\protect\hyperlink{ref-GilardiEtAl2021}{GILARDI 等, 2021} ; \protect\hyperlink{ref-Rogers2003}{ROGERS, 2003} ; \protect\hyperlink{ref-Karch2007}{KARCH, 2007})。随着政策扩散研究的进一步深入,学者们对政策扩散理解的注重点``首次采纳''拓展至``政策过程'',更多的从内外部行为者之间的互动过程角度理解政策扩散。在这一互动过程中,沟通交流和组织对政策扩散的影响尤为重要。。\protect\hyperlink{ref-Rogers2003}{ROGERS} (\protect\hyperlink{ref-Rogers2003}{2003}) 将政策扩散从首次采纳,定义为``互动-采纳-治理''的政策过程中的创新扩散。进一步的,\protect\hyperlink{ref-GilardiEtAl2021}{GILARDI 等} (\protect\hyperlink{ref-GilardiEtAl2021}{2021}) 从问题定义和议程设置视角定义政策扩散,将政策扩散的研究再次拓展到政策制定的整个过程。

\textbf{政策结果视角。}对如何扩散的过度关注是政策扩散研究受到的主要批评之一。作为对这一批评的回应,学者开始从结果角度理解政策扩散。这一方面的研究主要包括政策趋同(Policy Convergence)和政策再造(Policy Reinvention)两个方面。\protect\hyperlink{ref-Berry1994}{BERRY} (\protect\hyperlink{ref-Berry1994}{1994}) 将政策扩散的过程定义为在不同的地理空间,某一方面的政策的相似性增加。\protect\hyperlink{ref-Inkeles2019}{INKELES} (\protect\hyperlink{ref-Inkeles2019}{2019}) 将政策扩散定义为政府政策从不同的位置,人为的变化到某些同一位置。除了政策趋同之外,学者们还从政策扩散的差异结果研究政策扩散,并提出政策再造的概念(\protect\hyperlink{ref-Clark1985}{CLARK, 1985})。政策扩散并不一定会导致不同部门政策的完全相同,政策扩散的对象即内部行动者也不是完全被动的接受政策扩散,反而会根据自身的实际情况对政策进行批判性接受(\protect\hyperlink{ref-GlickHays1991}{GLICK 等, 1991}; \protect\hyperlink{ref-Hays1996}{HAYS, 1996}; \protect\hyperlink{ref-MooneyLee1995}{MOONEY 等, 1995})。还有学者从政策执行的角度关注政策扩散,认为政策扩散是指某一政府部门的政策影响到其他政府部门的治理过程。(\protect\hyperlink{ref-Evans2009}{EVANS, 2009})

\textbf{全球化的视角}虽然政策扩散的概念起始于美国政治研究(\protect\hyperlink{ref-Walker1969}{WALKER, 1969}),但是受到比较政治学和国际政治领域学者的关注。(\protect\hyperlink{ref-Milner1998}{MILNER, 1998})国际关系和比较政治的学者们将政策扩散从国内政治领域拓展到国际政治层面。与比较政治文献一样,国际关系学者一直关注趋同,但国际关系学者更加注重国际组织在促进各国实现相似政策方面的作用,尤其是关于规范的扩散。这些研究深入考察了社会化过程和身份政治如何影响规范在国际社会的传播,尽管这些概念与美国地方政治有关,但在国内政策扩散的研究中并未探讨这些概念。(\protect\hyperlink{ref-Checkel1999}{CHECKEL, 1999})

根据上述梳理,我们可以发现,政策扩散的定义随着研究的不断深入而越来越丰富。从政策扩散的主体来看,从政府拓展到各类包括国际组织、社团等各类政治主体;从政策扩散的过程来看,从单一的政策首次接纳,扩展到``议程设置-政策采用-政策执行-政策结果''的整个过程;从政策扩散的内容来看,政策扩散从单一的国内成文政策扩展到全球政策和规范;从研究领域来看,政策扩散最早由美国国内政治的研究者提出后,迅速的被比较政治学、国际关系等领域的学者接受和借用。

\begin{figure}
\includegraphics[width=1\linewidth]{../figures/ngram} \caption{政策扩散的定义}\label{fig:unnamed-chunk-2}
\end{figure}

\hypertarget{ux6982ux5ff5ux6f14ux53d8}{%
\subsection{概念演变}\label{ux6982ux5ff5ux6f14ux53d8}}

\begin{figure}
\includegraphics[width=1\linewidth]{../figures/政策扩散的概念演变} \caption{政策扩散的概念演变}\label{fig:unnamed-chunk-3}
\end{figure}

\hypertarget{ux4e34ux8fd1ux6982ux5ff5ux8fa8ux6790}{%
\subsection{临近概念辨析}\label{ux4e34ux8fd1ux6982ux5ff5ux8fa8ux6790}}

厘清政策扩散和政策传播其他临近概念的联系和差别,有助于我们进一步理解政策扩散概念的内涵和特点。
政策是政治科学研究重要的研究对象和数据来源,政府间公共政策的相互传播和比较分析是一个重要的研究领域,在政治学中有着悠久的传统。该研究领域的主要争论之一集中在不同国家是否以及为何随着时间的推移制定类似政策的问题上,这一比较包含了不同层级、不同部门甚至是不同国家之间公共政策的比较。作为公共政策领域的核心议题,``政策扩散''、``政策转移''、``政策趋同''、``政策再造''等概念都是学者对``政策从一个政府传递到另一个政府''(\protect\hyperlink{ref-ShipanVolden2012}{SHIPAN 等, 2012})这一现象的描述和解释,它们的内涵、关注重点、研究进路等特点各异,但也存在着联系,在公共政策比较分析领域的知识积累上相互补充。

从经典定义上来看,``政策扩散''是指一项政策创新通过某种做法或实践,经过空间或时间的过程在政治系统成员中传播交流的过程,``扩散是一种交流新观念的交流''(\protect\hyperlink{ref-Rogers2003}{ROGERS, 2003}; \protect\hyperlink{ref-STRANG1991}{STRANG, 1991});``政策转移''是指存在于某一时间与空间的政策安排被用于另一时间和空间的政策设计(\protect\hyperlink{ref-Bennett1991}{BENNETT, 1991}),和它相似的概念还包括,吸取教训(Lesson Drawing)和政策模仿(Policy Emulation)。``政策趋同''是指在社会发展的相似条件下,政策在组织结构、过程和运行方面的相似性(\protect\hyperlink{ref-Knill2005}{KNILL, 2005})。这些相似主要包括政策目标、政策内容、政策工具、政策结果和政策风格五个方面的趋同。(\protect\hyperlink{ref-HolzingerKnill2005}{HOLZINGER 等, 2005});``政策再造''是指政策文本随着政策扩散的进程而变化和重新设计(\protect\hyperlink{ref-GlickHays1991}{GLICK 等, 1991}),一些政策的整合性会随着传播而扩大(\protect\hyperlink{ref-Hays1996}{HAYS, 1996}),另一些政策会由于其本身的特点而被其他政府修改(\protect\hyperlink{ref-Volden2006}{VOLDEN, 2006})。

``政策扩散''、``政策趋同''、``政策转移''和``政策再造''既有联系,又有本质差别:

首先,从关注重点上,这些概念各自侧重描述不同的政策传播的不同重点,``政策扩散''早期的研究更多关注的是政策传播的条件,即回答``政策传播''何以发生?学者从时间、空间和条件等方面的相似性上给出解释;``政策转移''是指侧重于政策转播的过程,它将政策从一个部门传播到另一个部门视作一个互动的阶段,而非``传染''的结果(\protect\hyperlink{ref-BensonJordan2011}{BENSON 等, 2011});但无论是``政策扩散''还是``政策转移''都关心的是政策制定过程中的传播,而忽视了这种传播的影响和结果。(\protect\hyperlink{ref-Ladi2011}{LADI, 2011})面对这些批评,学者提出了``政策趋同''和``政策再造''两个截然不同的解释,``政策趋同''描述的是政策传播后的相似趋势,原始政策某项政策的某个或多个特征,在一段时间内发展为更多政治实体的相似性。(\protect\hyperlink{ref-Berry1994}{BERRY, 1994})这一概念本身就是结果性而不是结论性的,它并像政策扩散和政策转移那样强调政策跟随者的主动性:``政策再造''弥补了政策扩散对政策内容缺乏关注的不足,主张政策再造的学者认为,政策传播不是复制的过程,政策传播的内部行为者的主动性会使得他们对政策传播产生抵制或修改的能动行为,创造出新的政策。(\protect\hyperlink{ref-GlickHays1991}{GLICK 等, 1991}; \protect\hyperlink{ref-MooneyLee1995}{MOONEY 等, 1995})

其次,从基本类型上,``政策扩散''领域的研究用``全国互动型''、``区域扩散型''(\protect\hyperlink{ref-HeichelEtAl2005}{HEICHEL 等, 2005})、``领导-跟进型''(\protect\hyperlink{ref-GrahamEtAl2008}{GRAHAM 等, 2008})和``垂直影响型''(\protect\hyperlink{ref-Heinze2011}{HEINZE, 2011})对现有的政策扩散进行分类,这一分类标准体现出对扩散条件明显的关注,无论是``全国互动型''的官员交流、``区域扩散型''的地理临近、``领导-跟进型''的``政策绩效'',还是``垂直影响型''的中央动力都侧重回答``政策传播''何以发生这一问题。``政策转移''从原始政策对内部行为者的影响程度出发, 将政策转移区分为``复制''、``效仿''、``混合''和``启发''四种类型,带有明显的过程色彩(\protect\hyperlink{ref-DolowitzMarsh2000}{DOLOWITZ 等, 2000});``σ-趋同、β-趋同、γ-趋同和δ-趋同''是学者根据不同政治实体间的政策距离进行的类别划分,体现出政策传播结果上的相似。虽然政策再造还未有典型的类型划分,但是现有拒绝型和修改型均体现出内部行动者在政策传播结果上的能动性。

\hypertarget{ux653fux7b56ux6269ux6563ux6982ux5ff5ux7684ux610fux4e49}{%
\subsubsection{政策扩散概念的意义}\label{ux653fux7b56ux6269ux6563ux6982ux5ff5ux7684ux610fux4e49}}

\textbf{概念全局性。}虽然早期的政策扩散研究注重于扩散条件的研究,但近年来,政策扩散研究扩散到了``政策文本''{[}@{]}、``政策行动者''{[}@{]}、``政策扩散条件''{[}@{]}、``政策扩散机制''{[}@{]}等整个政策传播的各个方面(\protect\hyperlink{ref-GrahamEtAl2013}{GRAHAM 等, 2013}),具有全局性的特征。随着研究的深入``政策扩散''概念指的不仅仅是类似政策的地理聚类,它是包含政府相互竞争、相互学习的整体政策传播过程。(\protect\hyperlink{ref-ShipanVolden2012}{SHIPAN 等, 2012})

\textbf{领域整合性。}正是由于政策扩散概念的全局性,其对政策传播中的其他概念也有着重要的整合功能,``政策转移''可以被视作``政策扩散''的过程,``政策扩散''的结果可以较好的由``政策趋同''和``政策再造概括''。

\textbf{研究综合性。}政策扩散的全局性不仅体现在全政策过程,还体现在政策扩散研究的综合性,由于扩散政策和扩散主体性质的不同,政策扩散收到包括``比较政治''、``国内政治''、``国际关系''、``公共行政''等政治科学各领域研究者的普遍关注。而且正如下文所说的,政策扩散研究在这些学科中不是孤立存在的,而是相互影响相互促进的。

\textbf{视角立体性。}早起的政策扩散研究关注时间、地点临近行对政策传播的影响。随着研究的深入,政策扩散的研究者开始不仅限于一时一地的扩散,而是将政策扩散研究视作一个兼具时效性和长期性的纵贯过程。一方面,政策扩散研究仍然关注单个政策在集中截面时间的政策扩散,另一方面,纵观的面板的政策研究开始增加,帮助学者更好的了解政策扩散的条件和机制。与此同时,政策行为体空间分布的差异,将政策扩散研究的视域拓展到新的纬度,从一个立体性的视角展现政策传播多维现实。

\hypertarget{ux6269ux6563ux4e3bux4f53ux8c01ux5f71ux54cdux6269ux6563}{%
\section{扩散主体:谁影响扩散}\label{ux6269ux6563ux4e3bux4f53ux8c01ux5f71ux54cdux6269ux6563}}

政策扩散主要由政策制定和变迁作为主要表现形式,因此研究者往往会将政策视为一个独立于政策过程的结果,在统计数字或是地理空间中寻找相关性{[}@{]},但每一个政策都是由具有不同偏好、目标、能力等特点的人来选择和影响的。{[}@{]}笔者将影响政策扩散的行为者分成``内部行为者''和``外部行为者''两大类。内部行为者即政策扩散发生的政府行为者,他们的身份、偏好、目标、能力和政策环境各异,从而影响政策的扩散;政策扩散作为一个受外部影响的行为,原始的政策创新者作为政策扩散的使动者,对政策扩散的方向和形式具有着重要的作用;除此之外,智库、媒体、移民和政府间组织政策企业家等``政策企业家''也是政策扩散重要的外部行为者。

\hypertarget{ux5185ux90e8ux884cux4e3aux8005ux653fux7b56ux9009ux62e9ux8005}{%
\subsubsection{内部行为者:政策选择者}\label{ux5185ux90e8ux884cux4e3aux8005ux653fux7b56ux9009ux62e9ux8005}}

\textbf{政策扩散发生地的政府成员是政策扩散的内部行为者。}\href{mailto:正如@ShipanVolden2012简练地将政策扩散定义为}{\nolinkurl{正如@ShipanVolden2012简练地将政策扩散定义为}}``政府的政策选择受到其他政府选择的影响''那样,在自身政策选择中收到其他政府政策影响的内部行为者是理解政策扩散的核心。根据政策扩散发生的场域不同,地方政府{[}@{]}、不同政府部门{[}@{]}、主权国家{[}@{]}、国际组织{[}@{]}等等政治实体都可以成为政策扩散的内部行为者。他们的偏好、目标、能力、政策环境和身份各异,因此受到其他政府影响的程度和做出的反应也有较大差别,从而对政策扩散的过程和结果具有直接的影响。

\textbf{政策扩散内部行为者的偏好,是政策扩散的起点。}他们的偏好主要包括个人偏好和外界影响的偏好。他们的偏好可能由个人的学历{[}@{]}和经历{[}@{]}影响,也会受到民众{[}@{]}、利益集团{[}@{]}和其他政府的影响。

\textbf{偏好是政策扩散的起点,内部行为者的目标塑造者他们的政策偏好。}一般来说,作为政治人物,内部行为者的目标主要包括政治目标和治理目标两个大类。政治目标主要包括连任和提升合法性两种诉求。政治目标由内部行动者的合法性需求产生,他们需要借鉴别的政治实体的政策来显得更加有威望(\protect\hyperlink{ref-FordhamAsal2007}{FORDHAM 等, 2007})或治理社会的能力(\protect\hyperlink{ref-IkenberryKupchan1990}{IKENBERRY 等, 1990/ed})。受到政治目标影响的内部行为者往往会有选择地从他人的经验中学习(\protect\hyperlink{ref-Gilardi2010b}{GILARDI, 2010}),他们会根据先前使用者的意识形态来考量是否学习(\protect\hyperlink{ref-GrossbackEtAl2004a}{GROSSBACK 等, 2004})。政策目标主要包括扩大税基、增加收入等现实治理需要的满足, \protect\hyperlink{ref-Levi-Faur2003}{LEVI-FAUR} (\protect\hyperlink{ref-Levi-Faur2003}{2003}) 通过比较拉丁美洲和欧洲的政策扩散,发现有更加充足经济条件的政治实体往往会将政策目标视为主要诉求。

\textbf{偏好和目标影响了政策扩散的方向,但内部行为者的能力决定着政策扩散能否按照方向达成。} \protect\hyperlink{ref-ShipanVolden2006}{SHIPAN 等} (\protect\hyperlink{ref-ShipanVolden2006}{2006}) 通过美国禁烟政策扩散的考察发现,内部行为者的立法能力影响着政策的垂直扩散是否达成。由于能力较低,那些``不太专业''的州立法机构表现出强大的压力阀效应,即在所有真正需要禁烟的地方采取禁烟限制措施,减轻州政府采取行动的压力。在立法能力强的地方,更可能出现``滚雪球效应'',州立法者敏锐的发现到地方政策,并要求采取,将其扩展到全州,从而实现``民主实验室''的效果。在后续的研究中, \protect\hyperlink{ref-ShipanVolden2008}{SHIPAN 等} (\protect\hyperlink{ref-ShipanVolden2008}{2008}) 通过考察禁烟政策在城市间采用的异质性从自上而下和横向扩散两个层面进一步展现了地方政府能力对政策扩散的影响。地方能力强大的城市往往能够抵挡住中央政府政策压力,从而在政策扩散中具有更强的灵活性;地方能力弱小的城市往往会模仿中央或其他城市的政策,几乎这些政策不适合他们自己的社会。

\textbf{内部行为者的能力受到其政策决策的政治环境的限制}, \protect\hyperlink{ref-Weyland2005a}{WEYLAND} (\protect\hyperlink{ref-Weyland2005a}{2005}) 借助智利养老金私有化在拉丁美洲的扩散,发现时间和信息的有限影响着政策制定者的能力,从而影响着他们是采用学习、屈服于胁迫,或者根本不改变政策。 \protect\hyperlink{ref-Stone1999}{STONE} (\protect\hyperlink{ref-Stone1999}{1999}) 认为,面临经济危机或经历近期军事失败的政府更容易受到胁迫。 \protect\hyperlink{ref-BaileyRom2004}{BAILEY 等} (\protect\hyperlink{ref-BaileyRom2004}{2004}) 发现,原本福利水平更高的政府在其再分配政策中比那些已经低福利水平的政府更能应对竞争压力。选举连任的压力也一定程度上束缚着内部行为者能力的发挥,甚至降低其做出更优判断的能力。(\protect\hyperlink{ref-Karch2007a}{KARCH, 2007/ed})

\textbf{内部行为者的特点影响着政策扩散的方式和过程。}例如,内部参与者的特征似乎在政策以何种方式传播方面发挥着重要作用。\protect\hyperlink{ref-Fuglister2012}{FÜGLISTER} (\protect\hyperlink{ref-Fuglister2012}{2012}) 发现政府间机构的资格与卫生政策的扩散效果十分相关。(\protect\hyperlink{ref-Milner2006ux53d1ux73b0ux5a01ux6743ux653fux5e9cux548cux6c11ux4e3bux653fux5e9cux76f8ux6bd4ux66f4ux4e0dux592aux63a5ux53d7ux6280ux672fux521bux65b0ux7684ux4f20ux64ad}{\textbf{Milner2006发现威权政府和民主政府相比更不太接受技术创新的传播?}})。

\textbf{对于内部行为者来说,政策扩散并非单向接受的过程,而是一个主动选择的过程。}内部行动者在接受政策扩散的过程当中并不是完全被动的(\protect\hyperlink{ref-GlickHays1991}{GLICK 等, 1991}),因为无论是政策还是规范的都是可以背拒绝或者至少是修改的,因此,内部行动者会根据其自身的偏好和目标和外部行为者在政策上进行博弈,并根据其自身需要进行``政策再造'' ,(\protect\hyperlink{ref-Hays1996}{HAYS, 1996})以获得最有力的政策效果。与此同时,值得强调的是,不同政府同外部执行者博弈的空间并不相同,能力越强的政府对于政策修改的空间越大。{[}@{]}

\hypertarget{ux5916ux90e8ux4e3bux4f53}{%
\subsubsection{外部主体}\label{ux5916ux90e8ux4e3bux4f53}}

\textbf{政策扩散的外部行动者主要包括其他政治实体和``政策企业家''两大类。}其他政治实体是指已经采取了政策的行动者,他们可以是最早的政策创新者,也可以是上一次政策扩散是的政策接受者。{[}@{]}对于水平扩散来说,这里的外部政治实体是指和政策扩散对象一致的同级政府,他们的特点影响着内部参与者的政策扩散行为, (\protect\hyperlink{ref-Pacheco2012}{\textbf{Pacheco2012?}}) 发现外部行动者的性质特点和随之而来的政治态度对于影响其他政府是否效仿政策十分重要。例如,拥有更多专业知识的政府可能被视为领导者,更有可能向未来的采用者提供信息。因为潜在的采用者可能更有可能模仿大型或富裕的政府。同级政府的外界行为者往往会采取主动行动促进政策扩散,面对政策接受者的不同反应,他们会用不同的方式进行主动互动,如果面对竞争,他们会主动采取策略来使自己的原始政策更加有竞争力(\protect\hyperlink{ref-BaybeckEtAl2011}{BAYBECK 等, 2011});如果扩散对象采用学习的方式,那同级政府会借助这个机会进一步传播自己的政策标准,从而在之后的政策过程中能够扮演更加重要的角色;在社会化的过程中,他们更加有动机来增加主动传播,甚至他们的政治抱负本身就取决于向他人展示其政策的成功(\protect\hyperlink{ref-Adler1992}{ADLER, 1992/ed})

\textbf{对于垂直传播来说,根据垂直扩散的方向,外部政治实体可以是上级政府或是下级政府。}上级政府往往会使用胡萝卜加大棒的方式来推动政策的扩散。上级政府主要包括两种类型,一个是国家体制内的上级政府,更多的是中央政府,由于其拥有更多的行政手段,他们往往会使用行政压力来``胁迫''下级政府某种类似的政策{[}@{]};另一类是超国家组织,由于强制性不足,它们更多的采用信息提供或是经济激励来促进国际规范的扩散和传播\protect\hyperlink{ref-Drezner2005}{DREZNER} (\protect\hyperlink{ref-Drezner2005}{2005})
自下而上的垂直扩散,外部行为者是作为民主实验室的地方政府,它们将符合自身特殊实践的政策复制到更高的层级的政策中去,可以是从地区到州(\protect\hyperlink{ref-ShipanVolden2006}{SHIPAN 等, 2006}),也可以是从州到全国(\protect\hyperlink{ref-Boeckelman1992}{BOECKELMAN, 1992}),甚至也可以是从国家到超国家政府(\protect\hyperlink{ref-Drezner2005}{DREZNER, 2005})。

\textbf{政策企业家是近年来公共行政领域研究政策创新和政策扩散的重要主体。}简单来说,``政策企业家''就是指``利用自己自身资源来传播公共政策从而改变公共资源分配的外部行为者''(\protect\hyperlink{ref-Burgelman1985}{BURGELMAN, 1985})
现有研究主要关注``谁是政策企业家''、``哪些因素会影响政策企业家参与政策扩散''和``政策企业家如何参与政策扩散''三个领域。(\protect\hyperlink{ref-ZhuYaPengXiaoDiWen2014}{朱亚鹏 等, 2014})
\textbf{政策企业家主要包括政府间组织和政府外组织两种类型。}政府间的政策企业家往往由区域合作组织扮演,现有研究发现制度化的政府合作促进了福利(\protect\hyperlink{ref-Brooks2005}{BROOKS, 2005})、环保(\protect\hyperlink{ref-WardCao2012}{\textbf{WardCao2012?}})和卫生政策在政府间的传播{[}@{]};政府外的政策企业家可以是智库等学术机构(\protect\hyperlink{ref-Stone2004}{STONE, 2004})、媒体(\protect\hyperlink{ref-Dolowitz1997}{DOLOWITZ, 1997})、志愿团体(\protect\hyperlink{ref-SkocpolEtAl1993}{SKOCPOL 等, 1993})或者具有特定身份的公民群体(\protect\hyperlink{ref-Perez-ArmendarizCrow2010}{PÉREZ-ARMENDÁRIZ 等, 2010})
\textbf{政策企业家推动政策扩散收到多种条件的影响。}政策企业家所在组织的实力(\protect\hyperlink{ref-Schneider1989}{SCHNEIDER, 1989})、自身素质(\protect\hyperlink{ref-KingdonStano1984a}{KINGDON 等, 1984})、政治联盟(\protect\hyperlink{ref-DoigHargrove1990}{DOIG 等, 1990})、受到的激励(\protect\hyperlink{ref-Teodoro2009}{TEODORO, 2009})等等因素都会影响到政策企业家能否参与并推动政策创新。
\textbf{政策企业家在政策扩散的全过程都会起到推动或阻碍作用。}在议程设置阶段,他们会根据自己的目标界定政策议题的实质(\protect\hyperlink{ref-BaezAbolafia2002}{BAEZ 等, 2002});在推广政策阶段,为了突出自身政策的优势,他们会形成政治联盟进行游说,在政策窗口来临前为自身的政策积蓄力量(\protect\hyperlink{ref-MintromVergari1996}{MINTROM 等, 1996});面对同一治理问题,有时会有多种政策创新并行扩散,政策企业家为争取证明自己的政策方案的可行性,会进一步利用政治联盟来突出自己的方案或是贬低其他方案。(\protect\hyperlink{ref-Teodoro2009}{TEODORO, 2009})

\hypertarget{ux6269ux6563ux5185ux5bb9}{%
\subsection{扩散内容}\label{ux6269ux6563ux5185ux5bb9}}

``扩散''领域关注的扩散客体是丰富而广泛的{[}@{]}本文特别关注的是政治科学领域的扩散,他们包括常规政策的扩散,不仅是政策文本,现有研究还关注了政策扩散的议程设置和多次扩散等方面;政治科学范畴内公共政策之外的政策扩散也为我们理解政策扩散提供了有益的启示。

\hypertarget{ux5e38ux89c4ux653fux7b56ux6269ux6563}{%
\subsubsection{常规政策扩散}\label{ux5e38ux89c4ux653fux7b56ux6269ux6563}}

早期的政策扩散更加关注创新政策采用本身, \protect\hyperlink{ref-Walker1969}{WALKER} (\protect\hyperlink{ref-Walker1969}{1969}) 政策扩散开创性的研究就列举了八十八条测量创新分值的政策, \protect\hyperlink{ref-Gray1973}{GRAY} (\protect\hyperlink{ref-Gray1973}{1973}) 又将扩散的政策分为``教育、福利和公民权利''三个大的类别。后近的研究关注了性别平权政策、环境标准或是彩票等。这些虽然拓展了政策扩散的研究范围,但都没有突破原有的经典分类。这与这些政策的特点较为相关,政策的复杂性、可实验性、可复制性、可视化程度均会影响到它是否以及如何扩散。(\protect\hyperlink{ref-MakseVolden2011}{MAKSE 等, 2011})

政策采用实际上并不是一蹴而就的过程,实际上公共政策扩散往往会经过多次采用,每次采用也都伴随着政策接收对象的修改甚至是重造(\protect\hyperlink{ref-Clark1985}{CLARK, 1985})。政策随着扩散的的方式而演变和重新设计,比较同一政策的多次扩散(\protect\hyperlink{ref-MooneyLee1995}{MOONEY 等, 1995})、一次扩散的多次再造(\protect\hyperlink{ref-Hays1996}{HAYS, 1996})在文本上的不同,而不是将政策扩散视作单次的,拒绝或接受二分的过程,更有助于我们理解政策扩散的内部机制。

政策是由制定者观察和选择的,因此政策扩散的客体政策不仅和政策本身有关,其实际上应当是完整的政策过程各个阶段。大多数关于政策扩散的研究------即一个政府的政策制定影响其他政府的政策制定的过程------都集中在政策的采用上。然而,如果政策确实扩散,它们不会直接从一个地方的采用传播到另一个地方的采用 \protect\hyperlink{ref-GilardiEtAl2021}{GILARDI 等} (\protect\hyperlink{ref-GilardiEtAl2021}{2021}) 将重点转移到这个过程中一个重要但被忽视的方面:问题定义阶段,即政策过程的开始之处。

\hypertarget{ux975eux5e38ux89c4ux653fux7b56ux7684ux6269ux6563}{%
\subsubsection{非常规政策的扩散}\label{ux975eux5e38ux89c4ux653fux7b56ux7684ux6269ux6563}}

在政治科学领域中,还有诸如制度扩散、体制扩散、骚乱和政变的扩散等非常规的政策扩散。这类扩散虽不是我们关注的重点,但为我们理解政治扩散提供了有益的借鉴。制度扩散,尤其是民主扩散丰富的研究,为我们理解政治实体(尤其是国家实体)作出政治决策的内外部因素提出了重要的借鉴。(\protect\hyperlink{ref-LeesonDean2009}{LEESON 等, 2009}; \protect\hyperlink{ref-Starr1991a}{\textbf{Starr1991a?}}) 政府类型、机构改革(\protect\hyperlink{ref-BrinksCoppedge2006}{BRINKS 等, 2006})等体制结构上扩散为我们从组织层面理解政策扩散提供了视角;国际关系学者尤为关心的战争、骚乱(\protect\hyperlink{ref-LiThompson1975}{LI 等, 1975})和政变(\protect\hyperlink{ref-HillRothchild1986}{HILL 等, 1986})的扩散,直接从扩散客体上清楚的告诉我们:政治扩散并非都是有益的。(\protect\hyperlink{ref-ShipanVolden2012}{SHIPAN 等, 2012})

\hypertarget{ux6269ux6563ux903bux8f91}{%
\subsection{扩散逻辑}\label{ux6269ux6563ux903bux8f91}}

无论是政策问题的识别、设定、采用还是修改,都面临一个共同的问题:``一个政策为什么会从一个政治实体传播到另一个政治实体'',即政策扩散的逻辑。现有研究从临近性扩散(近邻性扩散)、策略性扩散(Strategic action)和相似导致扩散(Homophily)三个方面。

\hypertarget{ux8fd1ux90bbux6027ux6269ux6563proximity-diffusion}{%
\subsubsection{近邻性扩散(Proximity diffusion)}\label{ux8fd1ux90bbux6027ux6269ux6563proximity-diffusion}}

政策扩散的早期研究着眼于近邻性扩散, \protect\hyperlink{ref-ShipanVolden2012}{SHIPAN 等} (\protect\hyperlink{ref-ShipanVolden2012}{2012}) 将这一过程描述为``从落入池塘的鹅卵石中传播的涟漪''。政策扩散的这种近邻逻辑包括空间的近邻和时间的近邻, \protect\hyperlink{ref-Walker1969}{WALKER} (\protect\hyperlink{ref-Walker1969}{1969}) 对于政策扩散的开创性研究,认为是区域集群导致的政策扩散,这一政策扩散的经典理论一直延续到最近的研究。作为政策扩散良好的起点,并且在控制内部因素后,学者发现临近作用仍旧重要。但地理和空间的临近性往往过于局限,难以解释孤立的政策扩散现象。类似的政体,无论是否地理近邻,都可能采取类似的政策。
在当今世界,通信和旅行的障碍很低,将政策扩散作为地理集群的经典观点越来越过时。(\protect\hyperlink{ref-ShipanVolden2012}{SHIPAN 等, 2012})决策者在作出政策选择是会超越自己管辖的范畴,无论是时间还是空间。从空间上,上海不仅和北京竞争,还会和台北、首尔和多伦多竞争{[}@{]};从时间上,决策者不仅会采用临近的政策,还会从自身甚至别国的历史上寻找能够满足其治理或合法性需要的政策。{[}@{]}与此同时,近邻性扩散的机制也是模糊的,近邻性到底是如何导致政策扩散的,其是直接通过时间和空间的近邻本身提高政策制定者的可视程度?还是通过提供相似的政策环境,从而借助相似性扩散的机制实现政策扩散。

\hypertarget{ux76f8ux4f3cux6027ux6269ux6563homophily-diffusion}{%
\subsubsection{相似性扩散(Homophily diffusion)}\label{ux76f8ux4f3cux6027ux6269ux6563homophily-diffusion}}

比起近邻性扩散模糊的作用路径,相似性扩散提供了更加清楚的逻辑链条。决策者受到别的行动者政策选择影响是因为二者之间的相似性导致相似的政策选择预期。比起地理和时间的临近,宗教、语言、文化(\protect\hyperlink{ref-SimmonsElkins2004}{SIMMONS 等, 2004})、经济条件(\protect\hyperlink{ref-Volden2006}{VOLDEN, 2006})、意识形态(\protect\hyperlink{ref-GrossbackEtAl2004a}{GROSSBACK 等, 2004})等领域的相似\protect\hyperlink{ref-Berry1994}{BERRY} (\protect\hyperlink{ref-Berry1994}{1994})对政策扩散的机制提供了更加深入的理解。这种相似性不仅是行为者本身特点的相似,还包括政策环境和压力来源的相似。面对同样的政策窗口(\protect\hyperlink{ref-MooneyLee1999}{\textbf{MooneyLee1999?}})、同样的技术条件(\protect\hyperlink{ref-LeeEtAl2011}{\textbf{LeeEtAl2011?}})、同样的激励措施(\protect\hyperlink{ref-Mossberger2000}{\textbf{Mossberger2000?}})等,都可能是政策扩散的基础或条件。

\hypertarget{ux7b56ux7565ux6027ux6269ux6563strategic-diffusion}{%
\subsubsection{策略性扩散(Strategic diffusion)}\label{ux7b56ux7565ux6027ux6269ux6563strategic-diffusion}}

策略性扩散是指,政策扩散被内部行为者视作解决某种问题,或者面对某个治理挑战的回应工具。

\begin{quote}
``abandonment, acceptance, adaptation, adoption, amendment, avalanche, bandwagoning, best practices, billiard balls, borrowing, bottom-up, bubbling up, catalytic, change, clustering, coercion, communication, competition, contagion, cookie-cutter, co-operative, co-ordination, copying, convergence, cultural reference, decentralization, diffusion, divergence, disinhibition, emulation, enactment, experimentation, exporting, free-riding, Galton's problem, geographic, globalization, harmonization, hierarchical, horizontal, hybridization, imitation, importing, imposition, incentives, inducement, infection, innovation, insemination, inspiration, integration, interdependence, interstate, isomorphism, jumping, laboratories, laggards, leaders, leapfrogging, learning, lesson-drawing, linkages, localization, magnets, manipulation, mimicking, modelling, neighbours, networks, open method, peers, persuasion, pinching ideas, point source, pressure valve, prestige, problem solving, promotion, proneness, proximity, pruning, race to the bottom, reinforcement, reinvention, remodelling, S-curves, shaming, sharing, similarity, snowball, snowflakes, socialization, spatial, spread, success, synthesis, top-down, transfer, transitions, transnational, unification, vertical, voluntary, and whole-cloth.''
\end{quote}

\hypertarget{ux6269ux6563ux8defux5f84-ux5468ux4e94}{%
\subsection{扩散路径 (周五)}\label{ux6269ux6563ux8defux5f84-ux5468ux4e94}}

\hypertarget{ux5b66ux4e60ux673aux5236}{%
\subsubsection{学习机制}\label{ux5b66ux4e60ux673aux5236}}

\begin{itemize}
\item
  谁来学

  \begin{itemize}
  \item
    横向的学习
  \item
    纵向的学习
  \end{itemize}
\item
  学什么

  \begin{itemize}
  \item
    正向学习
  \item
    负向学习
  \end{itemize}
\item
  怎么学

  \begin{itemize}
  \item
    什么是成功的政策
  \item
    如何识别政策成功
  \end{itemize}
\end{itemize}

\hypertarget{ux7adeux4e89ux673aux5236}{%
\subsubsection{竞争机制}\label{ux7adeux4e89ux673aux5236}}

\begin{itemize}
\item
  怎么竞争
\item
  竞争的后果

  \begin{itemize}
  \item
    竞争的正向效应
  \item
    负向的负向效应
  \end{itemize}
\end{itemize}

\hypertarget{ux5f3aux5236ux673aux5236}{%
\subsubsection{强制机制}\label{ux5f3aux5236ux673aux5236}}

\begin{itemize}
\item
  谁强制,强制谁
\item
  怎么强制:强制的手段
\end{itemize}

\hypertarget{ux793eux4f1aux5316ux673aux5236}{%
\subsubsection{社会化机制}\label{ux793eux4f1aux5316ux673aux5236}}

\begin{itemize}
\item
  谁来社会化

  \begin{itemize}
  \item
    社会化的行动者
  \item
    不是单向的
  \end{itemize}
\item
  怎么社会化
\item
  社会化的效应(特点)
\end{itemize}

\hypertarget{ux591aux5143ux673aux5236}{%
\subsubsection{多元机制}\label{ux591aux5143ux673aux5236}}

\begin{itemize}
\item
  多种机制混合
\item
  补充而非替代
\end{itemize}

\hypertarget{ux653fux7b56ux6269ux6563ux65b9ux6cd5ux8defux5f84ux5468ux516d}{%
\subsection{政策扩散方法路径(周六)}\label{ux653fux7b56ux6269ux6563ux65b9ux6cd5ux8defux5f84ux5468ux516d}}

\hypertarget{ux73b0ux8c61ux5206ux6790}{%
\subsubsection{现象分析}\label{ux73b0ux8c61ux5206ux6790}}

\hypertarget{ux673aux5236ux5206ux6790}{%
\subsubsection{机制分析}\label{ux673aux5236ux5206ux6790}}

\hypertarget{ux6df7ux5408ux5206ux6790}{%
\subsubsection{混合分析}\label{ux6df7ux5408ux5206ux6790}}

\hypertarget{ux653fux7b56ux6269ux6563ux7814ux7a76ux7684ux5b66ux79d1ux8303ux5f0f}{%
\section{政策扩散研究的学科范式}\label{ux653fux7b56ux6269ux6563ux7814ux7a76ux7684ux5b66ux79d1ux8303ux5f0f}}

如前文所述,``政策扩散''受到政治科学各个领域学者的关注,形成了各具特色的学科范式,不同子领域即各自发展,又相互借鉴,共同构成了政策扩散研究的整体图景。现有研究主要包括国际关系、国内政治、比较政治和公共行政四个主要范式{[}@{]},梳理这些范式的区别和连接能够让我们更加完整的了解政策扩散的研究进度并整合多个领域的见解。

\hypertarget{ux6bd4ux8f83ux653fux6cbbux8303ux5f0f}{%
\subsection{比较政治范式}\label{ux6bd4ux8f83ux653fux6cbbux8303ux5f0f}}

\hypertarget{ux56fdux9645ux5173ux7cfbux8303ux5f0f}{%
\subsection{国际关系范式}\label{ux56fdux9645ux5173ux7cfbux8303ux5f0f}}

\hypertarget{ux56fdux5185ux653fux6cbbux8303ux5f0f}{%
\subsection{国内政治范式}\label{ux56fdux5185ux653fux6cbbux8303ux5f0f}}

\hypertarget{ux516cux5171ux884cux653fux5b66ux8303ux5f0f}{%
\subsection{公共行政学范式}\label{ux516cux5171ux884cux653fux5b66ux8303ux5f0f}}

\hypertarget{ux4e0dux540cux8303ux5f0fux7684ux6574ux5408ux548cux501fux9274}{%
\subsection{不同范式的整合和借鉴}\label{ux4e0dux540cux8303ux5f0fux7684ux6574ux5408ux548cux501fux9274}}

\hypertarget{ux73b0ux6709ux4e0dux8db3ux4e0eux76f2ux70b9ux8fd9ux4e00ux6574ux5757ux6574ux5408ux5ea6ux5f88ux4f4eux5468ux65e5}{%
\section{现有不足与盲点(这一整块整合度很低)(周日)}\label{ux73b0ux6709ux4e0dux8db3ux4e0eux76f2ux70b9ux8fd9ux4e00ux6574ux5757ux6574ux5408ux5ea6ux5f88ux4f4eux5468ux65e5}}

\hypertarget{ux89e3ux91caux7684ux4e0dux8db3}{%
\subsection{解释的不足}\label{ux89e3ux91caux7684ux4e0dux8db3}}

\hypertarget{ux8c01ux6269ux6563}{%
\subsubsection{谁扩散}\label{ux8c01ux6269ux6563}}

\begin{itemize}
\item
  内部参与者

  \begin{itemize}
  \item
    对决策者的联系关注不足(引出后文的干部交流)
  \item
    把决策者异质性关注的不足(引出后文的能力)
  \end{itemize}

  对能力的关注不足,尤其是地方政府能力的类别,哪些能力?治理能力?立法能力

  \begin{itemize}
  \tightlist
  \item
    对执行者关注的不足(引出后文的执行者)
  \end{itemize}
\item
  外部参与者

  \begin{itemize}
  \item
    对民众关注不足(引出后文的回应之回应)
  \item
    对全球化的关注不足(引出后文的国际组织直接影响地方政府)
  \end{itemize}
\item
  国际和地方研究的脱节
\end{itemize}

\hypertarget{ux6269ux6563ux4ec0ux4e48}{%
\subsubsection{扩散什么}\label{ux6269ux6563ux4ec0ux4e48}}

\begin{itemize}
\item
  政策扩散使用二分的观点,对文本变化的关注不足(引出后文的政策再造)
\item
  过于关注政策采用而非政策制定(引出议程设置中的政策扩散,``会议政治'')
\item
  过于关注首次采用而非一个多次的互动方式
\item
  对政策本身关注的不足
\item
  对政策执行中的政策扩散关注不足
\end{itemize}

\hypertarget{ux4e3aux4ec0ux4e48ux6269ux6563}{%
\subsubsection{为什么扩散}\label{ux4e3aux4ec0ux4e48ux6269ux6563}}

\begin{itemize}
\item
  对强制的自主性关注不足
\item
  对机制的整合不足
\end{itemize}

\hypertarget{ux600eux4e48ux6269ux6563}{%
\subsubsection{怎么扩散}\label{ux600eux4e48ux6269ux6563}}

\hypertarget{ux4f7fux7528ux7684ux4e0dux8db3}{%
\subsection{使用的不足}\label{ux4f7fux7528ux7684ux4e0dux8db3}}

\hypertarget{ux65b9ux6cd5ux7684ux4e0dux8db3}{%
\subsection{方法的不足}\label{ux65b9ux6cd5ux7684ux4e0dux8db3}}

权力下放和政策扩散

\hypertarget{ux672aux6765ux65b9ux5411ux57faux4e8eux653fux6cbbux5b66ux89c6ux89d2-ux8fd9ux4e09ux90e8ux5206ux90fdux770bux4e0dux51faux5bf9ux653fux6cbbux5b66ux7684ux91cdux8981ux6027}{%
\section{未来方向:基于政治学视角 (这三部分都看不出对政治学的重要性)}\label{ux672aux6765ux65b9ux5411ux57faux4e8eux653fux6cbbux5b66ux89c6ux89d2-ux8fd9ux4e09ux90e8ux5206ux90fdux770bux4e0dux51faux5bf9ux653fux6cbbux5b66ux7684ux91cdux8981ux6027}}

\hypertarget{ux4f5cux4e3aux56e0ux53d8ux91cfux7684ux653fux7b56ux6269ux6563}{%
\subsection{作为因变量的政策扩散}\label{ux4f5cux4e3aux56e0ux53d8ux91cfux7684ux653fux7b56ux6269ux6563}}

\hypertarget{ux56fdux5bb6ux80fdux529bux4e0eux653fux7b56ux6269ux6563ux7814ux7a76}{%
\subsubsection{国家能力与政策扩散研究}\label{ux56fdux5bb6ux80fdux529bux4e0eux653fux7b56ux6269ux6563ux7814ux7a76}}

\hypertarget{ux653fux5e9cux56deux5e94ux6027ux4e0eux653fux7b56ux6269ux6563ux7814ux7a76}{%
\subsubsection{政府回应性与政策扩散研究}\label{ux653fux5e9cux56deux5e94ux6027ux4e0eux653fux7b56ux6269ux6563ux7814ux7a76}}

\hypertarget{ux5168ux7403ux5316ux4e0eux653fux7b56ux6269ux6563ux7814ux7a76}{%
\subsubsection{全球化与政策扩散研究}\label{ux5168ux7403ux5316ux4e0eux653fux7b56ux6269ux6563ux7814ux7a76}}

\hypertarget{ux4f5cux4e3aux81eaux53d8ux91cfux7684ux653fux7b56ux6269ux6563}{%
\subsection{作为自变量的政策扩散}\label{ux4f5cux4e3aux81eaux53d8ux91cfux7684ux653fux7b56ux6269ux6563}}

\hypertarget{ux89c2ux5bdfux7eb5ux5411ux653fux5e9cux95f4ux5173ux7cfbux7684ux7a97ux53e3}{%
\subsubsection{观察纵向政府间关系的窗口}\label{ux89c2ux5bdfux7eb5ux5411ux653fux5e9cux95f4ux5173ux7cfbux7684ux7a97ux53e3}}

\hypertarget{ux89c2ux5bdfux6a2aux5411ux653fux5e9cux95f4ux5173ux7cfbux7684ux7a97ux53e3}{%
\subsubsection{观察横向政府间关系的窗口}\label{ux89c2ux5bdfux6a2aux5411ux653fux5e9cux95f4ux5173ux7cfbux7684ux7a97ux53e3}}

\hypertarget{ux89c2ux5bdfux56fdux5bb6ux793eux4f1aux5173ux7cfbux7684ux7a97ux53e3}{%
\subsubsection{观察国家社会关系的窗口}\label{ux89c2ux5bdfux56fdux5bb6ux793eux4f1aux5173ux7cfbux7684ux7a97ux53e3}}

\hypertarget{ux57faux4e8eux5927ux6570ux636eux65b9ux6cd5ux7684ux653fux7b56ux6269ux6563ux7814ux7a76}{%
\subsection{基于大数据方法的政策扩散研究}\label{ux57faux4e8eux5927ux6570ux636eux65b9ux6cd5ux7684ux653fux7b56ux6269ux6563ux7814ux7a76}}

\hypertarget{ux81eaux7136ux8bedux8a00ux5904ux7406ux65b9ux6cd5}{%
\subsubsection{自然语言处理方法}\label{ux81eaux7136ux8bedux8a00ux5904ux7406ux65b9ux6cd5}}

\hypertarget{ux793eux4ea4ux7f51ux7edcux5206ux6790ux65b9ux6cd5}{%
\subsubsection{社交网络分析方法}\label{ux793eux4ea4ux7f51ux7edcux5206ux6790ux65b9ux6cd5}}

\hypertarget{ux53c2ux8003ux6587ux732e}{%
\subsection{参考文献}\label{ux53c2ux8003ux6587ux732e}}

\hypertarget{ux9802ux5c64ux8a2dux8a08ux548cux653fux7b56ux8a66ux9ede}{%
\subsection{頂層設計和政策試點}\label{ux9802ux5c64ux8a2dux8a08ux548cux653fux7b56ux8a66ux9ede}}

能力,哪些能力
試點,
更多的关注政策创新
传统政策的扩散

\hypertarget{refs}{}
\begin{CSLReferences}{1}{0}
\leavevmode\vadjust pre{\hypertarget{ref-Adler1992}{}}%
ADLER E, 1992/ed. The Emergence of Cooperation: National Epistemic Communities and the International Evolution of the Idea of Nuclear Arms Control{[}J{]}. International Organization, 46(1): 101--145. DOI:\href{https://doi.org/10.1017/S0020818300001466}{10.1017/S0020818300001466}.

\leavevmode\vadjust pre{\hypertarget{ref-BaezAbolafia2002}{}}%
BAEZ B, ABOLAFIA M Y, 2002. Bureaucratic Entrepreneurship and Institutional Change: A Sense-Making Approach{[}J{]}. Journal of public administration research and theory, 12(4): 525--552.

\leavevmode\vadjust pre{\hypertarget{ref-BaileyRom2004}{}}%
BAILEY M A, ROM M C, 2004. A {Wider Race}? {Interstate Competition} across {Health} and {Welfare Programs}{[}J{]}. The Journal of Politics, 66(2): 326--347. DOI:\href{https://doi.org/10.1111/j.1468-2508.2004.00154.x}{10.1111/j.1468-2508.2004.00154.x}.

\leavevmode\vadjust pre{\hypertarget{ref-BaybeckEtAl2011}{}}%
BAYBECK B, BERRY W D, SIEGEL D A, 2011. A {Strategic Theory} of {Policy Diffusion} via {Intergovernmental Competition}{[}J{]}. The Journal of Politics, 73(1): 232--247. DOI:\href{https://doi.org/10.1017/S0022381610000988}{10.1017/S0022381610000988}.

\leavevmode\vadjust pre{\hypertarget{ref-Bennett1991}{}}%
BENNETT C J, 1991. What Is Policy Convergence and What Causes It?{[}J{]}. British journal of political science, 21(2): 215--233.

\leavevmode\vadjust pre{\hypertarget{ref-BensonJordan2011}{}}%
BENSON D, JORDAN A, 2011. What Have We {Learned} from {Policy Transfer Research}? {Dolowitz} and {Marsh Revisited}{[}J{]}. Political Studies Review, 9(3): 366--378. DOI:\href{https://doi.org/10.1111/j.1478-9302.2011.00240.x}{10.1111/j.1478-9302.2011.00240.x}.

\leavevmode\vadjust pre{\hypertarget{ref-Berry1994}{}}%
BERRY F S, 1994. Sizing up {State Policy Innovation Research}{[}J{]}. Policy Studies Journal, 22(3): 442--456.

\leavevmode\vadjust pre{\hypertarget{ref-Boeckelman1992}{}}%
BOECKELMAN K, 1992. The {Influence Of States On Federal Policy Adoptions}{[}J{]}. Policy Studies Journal, 20(3): 365--375. DOI:\href{https://doi.org/10.1111/j.1541-0072.1992.tb00164.x}{10.1111/j.1541-0072.1992.tb00164.x}.

\leavevmode\vadjust pre{\hypertarget{ref-BrinksCoppedge2006}{}}%
BRINKS D, COPPEDGE M, 2006. Diffusion {Is No Illusion}: {Neighbor Emulation} in the {Third Wave} of {Democracy}{[}J{]}. Comparative Political Studies, 39(4): 463--489. DOI:\href{https://doi.org/10.1177/0010414005276666}{10.1177/0010414005276666}.

\leavevmode\vadjust pre{\hypertarget{ref-Brooks2005}{}}%
BROOKS S M, 2005. Interdependent and {Domestic Foundations} of {Policy Change}: {The Diffusion} of {Pension Privatization Around} the {World}{[}J{]}. International Studies Quarterly, 49(2): 273--294. DOI:\href{https://doi.org/10.1111/j.0020-8833.2005.00345.x}{10.1111/j.0020-8833.2005.00345.x}.

\leavevmode\vadjust pre{\hypertarget{ref-Burgelman1985}{}}%
BURGELMAN R A, 1985. Review of {Public Entrepreneurship}: {Toward} a {Theory} of {Bureaucratic Political Power}.{[}J{]}. Administrative Science Quarterly, 30(4): 594--596. DOI:\href{https://doi.org/10.2307/2392700}{10.2307/2392700}.

\leavevmode\vadjust pre{\hypertarget{ref-Checkel1999}{}}%
CHECKEL J T, 1999. Norms, Institutions, and National Identity in Contemporary {Europe}{[}J{]}. International studies quarterly, 43(1): 83--114.

\leavevmode\vadjust pre{\hypertarget{ref-Clark1985}{}}%
CLARK K B, 1985. The Interaction of Design Hierarchies and Market Concepts in Technological Evolution{[}J{]}. Research Policy, 14(5): 235--251. DOI:\href{https://doi.org/10.1016/0048-7333(85)90007-1}{10.1016/0048-7333(85)90007-1}.

\leavevmode\vadjust pre{\hypertarget{ref-DoigHargrove1990}{}}%
DOIG J W, HARGROVE E C, 1990. Leadership and Innovation: {Entrepreneurs} in Government{[}M{]}. {JHU Press}.

\leavevmode\vadjust pre{\hypertarget{ref-Dolowitz1997}{}}%
DOLOWITZ D P, 1997. British {Employment Policy} in the 1980s: {Learning} from the {American Experience}{[}J{]}. Governance, 10(1): 23--42. DOI:\href{https://doi.org/10.1111/0952-1895.271996027}{10.1111/0952-1895.271996027}.

\leavevmode\vadjust pre{\hypertarget{ref-DolowitzMarsh2000}{}}%
DOLOWITZ D P, MARSH D, 2000. Learning from Abroad: {The} Role of Policy Transfer in Contemporary Policy-making{[}J{]}. Governance, 13(1): 5--23.

\leavevmode\vadjust pre{\hypertarget{ref-Drezner2005}{}}%
DREZNER D W, 2005. Globalization, Harmonization, and Competition: The Different Pathways to Policy Convergence{[}J{]}. Journal of European Public Policy, 12(5): 841--859. DOI:\href{https://doi.org/10.1080/13501760500161472}{10.1080/13501760500161472}.

\leavevmode\vadjust pre{\hypertarget{ref-Evans2009}{}}%
EVANS M, 2009. Policy Transfer in Critical Perspective{[}J{]}. Policy Studies, 30(3): 243--268. DOI:\href{https://doi.org/10.1080/01442870902863828}{10.1080/01442870902863828}.

\leavevmode\vadjust pre{\hypertarget{ref-FordhamAsal2007}{}}%
FORDHAM B O, ASAL V, 2007. Billiard {Balls} or {Snowflakes}? {Major Power Prestige} and the {International Diffusion} of {Institutions} and {Practices}{[}J{]}. International Studies Quarterly, 51(1): 31--52. DOI:\href{https://doi.org/10.1111/j.1468-2478.2007.00438.x}{10.1111/j.1468-2478.2007.00438.x}.

\leavevmode\vadjust pre{\hypertarget{ref-Fuglister2012}{}}%
FÜGLISTER K, 2012. Where Does Learning Take Place? {The} Role of Intergovernmental Cooperation in Policy Diffusion{[}J{]}. European Journal of Political Research, 51(3): 316--349. DOI:\href{https://doi.org/10.1111/j.1475-6765.2011.02000.x}{10.1111/j.1475-6765.2011.02000.x}.

\leavevmode\vadjust pre{\hypertarget{ref-Gilardi2010b}{}}%
GILARDI F, 2010. Who {Learns} from {What} in {Policy Diffusion Processes}?{[}J{]}. American Journal of Political Science, 54(3): 650--666. DOI:\href{https://doi.org/10.1111/j.1540-5907.2010.00452.x}{10.1111/j.1540-5907.2010.00452.x}.

\leavevmode\vadjust pre{\hypertarget{ref-GilardiEtAl2021}{}}%
GILARDI F, SHIPAN C R, WUEST B, 2021. Policy {Diffusion}: {The Issue-Definition Stage}{[}J{]}. AMERICAN JOURNAL OF POLITICAL SCIENCE, 65(1): 21--35. DOI:\href{https://doi.org/10.1111/ajps.12521}{10.1111/ajps.12521}.

\leavevmode\vadjust pre{\hypertarget{ref-GilardiWasserfallen2019}{}}%
GILARDI F, WASSERFALLEN F, 2019. The Politics of Policy Diffusion{[}J{]}. European Journal of Political Research, 58(4): 1245--1256. DOI:\href{https://doi.org/10.1111/1475-6765.12326}{10.1111/1475-6765.12326}.

\leavevmode\vadjust pre{\hypertarget{ref-GlickHays1991}{}}%
GLICK H R, HAYS S P, 1991. Innovation and {Reinvention} in {State Policymaking}: {Theory} and the {Evolution} of {Living Will Laws}{[}J{]}. The Journal of Politics, 53(3): 835--850. DOI:\href{https://doi.org/10.2307/2131581}{10.2307/2131581}.

\leavevmode\vadjust pre{\hypertarget{ref-GrahamEtAl2008}{}}%
GRAHAM E, SHIPAN C R, VOLDEN C, 2008. The Diffusion of Policy Diffusion Research{[}C{]}//Paper Presentation at the {Annual Meeting} of {APSA}. {Citeseer}.

\leavevmode\vadjust pre{\hypertarget{ref-GrahamEtAl2013}{}}%
GRAHAM E R, SHIPAN C R, VOLDEN C, 2013. The {Diffusion} of {Policy Diffusion Research} in {Political Science}{[}J{]}. British Journal of Political Science, 43(3): 673--701. DOI:\href{https://doi.org/10.1017/S0007123412000415}{10.1017/S0007123412000415}.

\leavevmode\vadjust pre{\hypertarget{ref-Gray1973}{}}%
GRAY V, 1973. Innovation in the {States}: {A Diffusion Study}*{[}J{]}. American Political Science Review, 67(4): 1174--1185. DOI:\href{https://doi.org/10.2307/1956539}{10.2307/1956539}.

\leavevmode\vadjust pre{\hypertarget{ref-GrossbackEtAl2004a}{}}%
GROSSBACK L J, NICHOLSON-CROTTY S, PETERSON D A M, 2004. Ideology and {Learning} in {Policy Diffusion}{[}J{]}. American Politics Research, 32(5): 521--545. DOI:\href{https://doi.org/10.1177/1532673X04263801}{10.1177/1532673X04263801}.

\leavevmode\vadjust pre{\hypertarget{ref-Hays1996}{}}%
HAYS S P, 1996. Patterns of {Reinvention}{[}J{]}. Policy Studies Journal, 24(4): 551--566. DOI:\href{https://doi.org/10.1111/j.1541-0072.1996.tb01646.x}{10.1111/j.1541-0072.1996.tb01646.x}.

\leavevmode\vadjust pre{\hypertarget{ref-HeichelEtAl2005}{}}%
HEICHEL S, PAPE J, SOMMERER T, 2005. Is There Convergence in Convergence Research? An Overview of Empirical Studies on Policy Convergence{[}J{]}. Journal of European Public Policy, 12(5): 817--840. DOI:\href{https://doi.org/10.1080/13501760500161431}{10.1080/13501760500161431}.

\leavevmode\vadjust pre{\hypertarget{ref-Heinze2011}{}}%
HEINZE T, 2011. Mechanism-Based Thinking on Policy Diffusion: A Review of Current Approaches in Political Science{[}J{]}. DOI:\href{https://doi.org/10.17169/refubium-22059}{10.17169/refubium-22059}.

\leavevmode\vadjust pre{\hypertarget{ref-HillRothchild1986}{}}%
HILL S, ROTHCHILD D, 1986. The {Contagion} of {Political Conflict} in {Africa} and the {World}{[}J{]}. Journal of Conflict Resolution, 30(4): 716--735. DOI:\href{https://doi.org/10.1177/0022002786030004006}{10.1177/0022002786030004006}.

\leavevmode\vadjust pre{\hypertarget{ref-HolzingerKnill2005}{}}%
HOLZINGER K, KNILL C, 2005. Causes and Conditions of Cross-National Policy Convergence{[}J{]}. Journal of European Public Policy, 12(5): 775--796. DOI:\href{https://doi.org/10.1080/13501760500161357}{10.1080/13501760500161357}.

\leavevmode\vadjust pre{\hypertarget{ref-IkenberryKupchan1990}{}}%
IKENBERRY G J, KUPCHAN C A, 1990/ed. Socialization and Hegemonic Power{[}J{]}. International Organization, 44(3): 283--315. DOI:\href{https://doi.org/10.1017/S002081830003530X}{10.1017/S002081830003530X}.

\leavevmode\vadjust pre{\hypertarget{ref-Inkeles2019}{}}%
INKELES A, 2019. One {World Emerging}? {Convergence And Divergence In Industrial Societies}{[}M{]}. {Routledge}.

\leavevmode\vadjust pre{\hypertarget{ref-Karch2007}{}}%
KARCH A, 2007. Democratic {Laboratories}: {Policy Diffusion Among} the {American States}{[}M{]}. {University of Michigan Press}.

\leavevmode\vadjust pre{\hypertarget{ref-Karch2007a}{}}%
KARCH A, 2007/ed. Emerging {Issues} and {Future Directions} in {State Policy Diffusion Research}{[}J{]}. State Politics \& Policy Quarterly, 7(1): 54--80. DOI:\href{https://doi.org/10.1177/153244000700700104}{10.1177/153244000700700104}.

\leavevmode\vadjust pre{\hypertarget{ref-KingdonStano1984a}{}}%
KINGDON J W, STANO E, 1984. Agendas, Alternatives, and Public Policies{[}M{]}. {Little, Brown Boston}.

\leavevmode\vadjust pre{\hypertarget{ref-Knill2005}{}}%
KNILL C, 2005. Introduction: {Cross-national} Policy Convergence: Concepts, Approaches and Explanatory Factors{[}J{]}. Journal of European Public Policy, 12(5): 764--774. DOI:\href{https://doi.org/10.1080/13501760500161332}{10.1080/13501760500161332}.

\leavevmode\vadjust pre{\hypertarget{ref-Ladi2011}{}}%
LADI S, 2011. Policy {Change} and {Soft Europeanization}: {The Transfer} of the {Ombudsman Institution} to {Greece}, {Cyprus} and {Malta}{[}J{]}. Public Administration, 89(4): 1643--1663. DOI:\href{https://doi.org/10.1111/j.1467-9299.2011.01929.x}{10.1111/j.1467-9299.2011.01929.x}.

\leavevmode\vadjust pre{\hypertarget{ref-LeesonDean2009}{}}%
LEESON P T, DEAN A M, 2009. The {Democratic Domino Theory}: {An Empirical Investigation}{[}J{]}. American Journal of Political Science, 53(3): 533--551. DOI:\href{https://doi.org/10.1111/j.1540-5907.2009.00385.x}{10.1111/j.1540-5907.2009.00385.x}.

\leavevmode\vadjust pre{\hypertarget{ref-Levi-Faur2003}{}}%
LEVI-FAUR D, 2003. The Politics of Liberalisation: {Privatisation} and Regulation-for-Competition in {Europe}'s and {Latin America}'s Telecoms and Electricity Industries{[}J{]}. European Journal of Political Research, 42(5): 705--740. DOI:\href{https://doi.org/10.1111/1475-6765.00101}{10.1111/1475-6765.00101}.

\leavevmode\vadjust pre{\hypertarget{ref-LiThompson1975}{}}%
LI R P Y, THOMPSON W R, 1975. The "{Coup Contagion}" {Hypothesis}{[}J{]}. Journal of Conflict Resolution, 19(1): 63--84. DOI:\href{https://doi.org/10.1177/002200277501900104}{10.1177/002200277501900104}.

\leavevmode\vadjust pre{\hypertarget{ref-Lucas1983}{}}%
LUCAS A, 1983. Public {Policy Diffusion Research}: {Integrating Analytic Paradigms}{[}J{]}. Knowledge, 4(3): 379--408. DOI:\href{https://doi.org/10.1177/107554708300400303}{10.1177/107554708300400303}.

\leavevmode\vadjust pre{\hypertarget{ref-MakseVolden2011}{}}%
MAKSE T, VOLDEN C, 2011. The {Role} of {Policy Attributes} in the {Diffusion} of {Innovations}{[}J{]}. JOURNAL OF POLITICS, 73(1): 108--124. DOI:\href{https://doi.org/10.1017/S0022381610000903}{10.1017/S0022381610000903}.

\leavevmode\vadjust pre{\hypertarget{ref-Milner1998}{}}%
MILNER H V, 1998. Rationalizing Politics: {The} Emerging Synthesis of International, {American}, and Comparative Politics{[}J{]}. International Organization, 52(4): 759--786.

\leavevmode\vadjust pre{\hypertarget{ref-MintromVergari1996}{}}%
MINTROM M, VERGARI S, 1996. Advocacy Coalitions, Policy Entrepreneurs, and Policy Change{[}J{]}. Policy studies journal, 24(3): 420--434.

\leavevmode\vadjust pre{\hypertarget{ref-MooneyLee1995}{}}%
MOONEY C Z, LEE M-H, 1995. Legislative {Morality} in the {American States}: {The Case} of {Pre-Roe Abortion Regulation Reform}{[}J{]}. American Journal of Political Science, 39(3): 599--627. DOI:\href{https://doi.org/10.2307/2111646}{10.2307/2111646}.

\leavevmode\vadjust pre{\hypertarget{ref-Perez-ArmendarizCrow2010}{}}%
PÉREZ-ARMENDÁRIZ C, CROW D, 2010. Do {Migrants Remit Democracy}? {International Migration}, {Political Beliefs}, and {Behavior} in {Mexico}{[}J{]}. Comparative Political Studies, 43(1): 119--148. DOI:\href{https://doi.org/10.1177/0010414009331733}{10.1177/0010414009331733}.

\leavevmode\vadjust pre{\hypertarget{ref-Rogers2003}{}}%
ROGERS E M, 2003. Diffusion of {Innovations}, 5th {Edition}{[}M{]}. 5th edition 版. {New York}: {Free Press}.

\leavevmode\vadjust pre{\hypertarget{ref-Schneider1989}{}}%
SCHNEIDER M, 1989. The Competitive City: {The} Political Economy of Suburbia{[}M{]}. {University of Pittsburgh Pre}.

\leavevmode\vadjust pre{\hypertarget{ref-ShipanVolden2006}{}}%
SHIPAN C R, VOLDEN C, 2006. Bottom-up Federalism: {The} Diffusion of Antismoking Policies from {US} Cities to States{[}J{]}. AMERICAN JOURNAL OF POLITICAL SCIENCE, 50(4): 825--843. DOI:\href{https://doi.org/10.1111/j.1540-5907.2006.00218.x}{10.1111/j.1540-5907.2006.00218.x}.

\leavevmode\vadjust pre{\hypertarget{ref-ShipanVolden2008}{}}%
SHIPAN C R, VOLDEN C, 2008. The {Mechanisms} of {Policy Diffusion}{[}J{]}. American Journal of Political Science, 52(4): 840--857. DOI:\href{https://doi.org/10.1111/j.1540-5907.2008.00346.x}{10.1111/j.1540-5907.2008.00346.x}.

\leavevmode\vadjust pre{\hypertarget{ref-ShipanVolden2012}{}}%
SHIPAN C R, VOLDEN C, 2012. Policy {Diffusion}: {Seven Lessons} for {Scholars} and {Practitioners}{[}J{]}. Public Administration Review, 72(6): 788--796. DOI:\href{https://doi.org/10.1111/j.1540-6210.2012.02610.x}{10.1111/j.1540-6210.2012.02610.x}.

\leavevmode\vadjust pre{\hypertarget{ref-SimmonsElkins2004}{}}%
SIMMONS B, ELKINS Z, 2004. The Globalization of Liberalization: {Policy} Diffusion in the International Political Economy{[}J{]}. AMERICAN POLITICAL SCIENCE REVIEW, 98(1): 171--189. DOI:\href{https://doi.org/10.1017/S0003055404001078}{10.1017/S0003055404001078}.

\leavevmode\vadjust pre{\hypertarget{ref-SkocpolEtAl1993}{}}%
SKOCPOL T, ABEND-WEIN M, HOWARD C, 等, 1993. Women's {Associations} and the {Enactment} of {Mothers}' {Pensions} in the {United States}{[}J{]}. American Political Science Review, 87(3): 686--701. DOI:\href{https://doi.org/10.2307/2938744}{10.2307/2938744}.

\leavevmode\vadjust pre{\hypertarget{ref-Stone1999}{}}%
STONE D, 1999. Learning {Lessons} and {Transferring Policy} Across {Time}, {Space} and {Disciplines}{[}J{]}. Politics, 19(1): 51--59. DOI:\href{https://doi.org/10.1111/1467-9256.00086}{10.1111/1467-9256.00086}.

\leavevmode\vadjust pre{\hypertarget{ref-Stone2004}{}}%
STONE D, 2004. Transfer Agents and Global Networks in the {〈Transnationalization〉} of Policy{[}J{]}. Journal of European Public Policy, 11(3): 545--566. DOI:\href{https://doi.org/10.1080/13501760410001694291}{10.1080/13501760410001694291}.

\leavevmode\vadjust pre{\hypertarget{ref-STRANG1991}{}}%
STRANG D, 1991. Adding {Social Structure} to {Diffusion Models}: {An Event History Framework}{[}J{]}. Sociological Methods \& Research, 19(3): 324--353. DOI:\href{https://doi.org/10.1177/0049124191019003003}{10.1177/0049124191019003003}.

\leavevmode\vadjust pre{\hypertarget{ref-Teodoro2009}{}}%
TEODORO M P, 2009. Bureaucratic {Job Mobility} and {The Diffusion} of {Innovations}{[}J{]}. AMERICAN JOURNAL OF POLITICAL SCIENCE, 53(1): 175--189. DOI:\href{https://doi.org/10.1111/j.1540-5907.2008.00364.x}{10.1111/j.1540-5907.2008.00364.x}.

\leavevmode\vadjust pre{\hypertarget{ref-Volden2006}{}}%
VOLDEN C, 2006. States as {Policy Laboratories}: {Emulating Success} in the {Children}'s {Health Insurance Program}{[}J{]}. American Journal of Political Science, 50(2): 294--312.

\leavevmode\vadjust pre{\hypertarget{ref-VoldenEtAl2008}{}}%
VOLDEN C, TING M M, CARPENTER D P, 2008. A Formal Model of Learning and Policy Diffusion{[}J{]}. AMERICAN POLITICAL SCIENCE REVIEW, 102(3): 319--332. DOI:\href{https://doi.org/10.1017/S0003055408080271}{10.1017/S0003055408080271}.

\leavevmode\vadjust pre{\hypertarget{ref-Walker1969}{}}%
WALKER J L, 1969. The {Diffusion} of {Innovations} Among the {American States}{[}J{]}. American Political Science Review, 63(3): 880--899. DOI:\href{https://doi.org/10.1017/S0003055400258644}{10.1017/S0003055400258644}.

\leavevmode\vadjust pre{\hypertarget{ref-WelchThompson1980}{}}%
WELCH S, THOMPSON K, 1980. The {Impact} of {Federal Incentives} on {State Policy Innovation}{[}J{]}. American Journal of Political Science, 24(4): 715--729. DOI:\href{https://doi.org/10.2307/2110955}{10.2307/2110955}.

\leavevmode\vadjust pre{\hypertarget{ref-Weyland2005a}{}}%
WEYLAND K, 2005. Theories of {Policy Diffusion Lessons} from {Latin American Pension Reform}{[}J{]}. World Politics, 57(2): 262--295. DOI:\href{https://doi.org/10.1353/wp.2005.0019}{10.1353/wp.2005.0019}.

\leavevmode\vadjust pre{\hypertarget{ref-ZhuYaPengXiaoDiWen2014}{}}%
朱亚鹏, 肖棣文, 2014. {政策企业家与社会政策创新}{[}J{]}. 社会学研究, 29(03): 56--76+242.

\leavevmode\vadjust pre{\hypertarget{ref-BaoWeiHui2021}{}}%
鲍伟慧, 2021. {政策扩散理论国外研究述评:态势、关注与展望}{[}J{]}. 内蒙古大学学报(哲学社会科学版), 53(04): 82--89.

\end{CSLReferences}

\end{document}
